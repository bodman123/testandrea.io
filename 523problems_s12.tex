\magnification=1100
\input amstex
\documentstyle{amsppt}
\pageheight{22truecm}
\pagewidth{15.6truecm}
\vcorrection{0.3truecm}
\hcorrection{0.4truecm}

\documentstyle{amsppt}
\NoBlackBoxes


%\NoPageNumbers
\def \IR {\Bbb R}
\def \IN {\Bbb N}
\def \IC {\Bbb C}
\def \B {\Cal B}
\def \M {\Cal M}
\def \N {\Cal N}
\def \H {\Cal H}
\def \C {\Cal C}
\def \R {\Cal R}
\def \A {\Cal A}
{\catcode`@=11
\gdef\nologo{\let\logo@\empty}
\catcode`@=12}
\nologo
%\hcorrection{.3in}
%\hskip 3in {\bf : }
%\vskip .3in
\centerline{ \bf M523Honors:  Introduction to Modern Analysis}
\vskip .1in
\centerline{ \bf  Homework}
\vskip .2in
\centerline{Spring 2012 }
\vskip .2in
\document

\underbar{{\bf Note:}}\quad  Do not turn in yet the \underbar{Special Projects} nor problems followed by an asterisk $*$. I'll announce when these are due.

\vskip .2in


\subhead{ Assignment 1.   Due  Thursday February 16th 2012}\endsubhead
 \vskip .2in 
 
\underbar{Extra Problem 1$^\ast$:}  \quad Show that $(\IC, + , \cdot)$ as defined
in class is a field.   
\vskip .1in 
\underbar{From Section 1.1:}  \quad 1, 3, 4, 5, 6a), 7, 10, 11, 12$*$
\vskip .2in


\underbar{From Section 1.2:} \quad 1, 2, 3, 5a), 5d), 6a), 6b), 6c), 7a),
7b), 7d), 9b), 10. 

\vskip .1in

\underbar{From Section 1.3:}\quad 1, 3, 5, 7, 8, 9*. 

\vskip .2in



\subhead{ Assignment 2.  Due Thursday February 23 2012 }\endsubhead
 \vskip .2in

\underbar{From Section 1.4:}\quad 1, 4, 7, 8, 9, 10a)c) 11a) 
\vskip .1in
(In 10c) $f^2$ means $[f(x)]^2$  and {\bf not} $f \circ f$ )   
\vskip .2in 

\underbar{From Section 2.1:}\quad 1, 2, 3, 4, 5, 7 



\subhead{ Assignment  3.  Due Thursday March 8th 2012}\endsubhead
\vskip .2in 

\underbar{From Section 2.2:}\, \, 1, 2, 3, 4, 5(modified) , 7, 8, 9.  
\vskip .1in 
{\bf Problem $5$(modified)}   Prove theorem 2.26 in the special case when
$a_n=1$. That is prove $\frac{1}{b_n}  \to \frac{1}{b}$ under the same
hypothesis.    I will post a handout for the general case later. 
\vskip .1in

\underbar{From Section 2.4:}\,  \, 1, 2, 3, 5, 6, 7, 9, 13. 

\vskip .2in

\underbar{Special Project I:}\,  Do Project $\#1$ at the end of Chapter 2. 



\subhead{ Assignment 4.   Due Thursday March 15th 2012 }\endsubhead

\vskip .1in

\underbar{From Section 1.3}  Problem 9 

\vskip .2in


\underbar{From Section 2.5:}\,\, 1, \, 3(modified), \, 4, 5, 7, 8.

\vskip .2in

\proclaim{ Problem 3(modified)}  Suppose a set $S$ of real numbers is
bounded and let $\eta$
be a lower bound for $S$. Show that $\eta$ is the {\it greatest lower bound} of
$S$ if and only if for every $\varepsilon >0$ there is an element of
$S$ in the interval $[\eta, \eta+\varepsilon]$
 \endproclaim

\newpage

\subhead{ Assignment 5.   Due Thursday March 29th 2012 }\endsubhead
\vskip .1in 

\underbar{Extra Problem 2$^\ast$} Prove the existence of {\it greatest lower bounds} just as we proved the 
existence of {\it least upper bound} in Theorem 2.5.1

\vskip .1in 

\underbar{From Section 2.6:}\,  \, 1, 2, 3, 4, 6,  8, 9, 10, 11, 13.

\vskip .1in


\underbar{Special Project II:}\,  Do Problem 14 of Section 2.4. 

\vskip .1in

\underbar{Special Project III:}\,  Do Project $\#5$ at the end of
Chapter 2 (page 70)



 \vskip .1in 


\subhead{ Assignment 6.  Due Thursday 4/12/2012  }\endsubhead
\vskip .1in 
\underbar{From Section 3.1 :}  2, 4, 5, 7, 8a)b), 9, 11. 

\vskip .1in


\vskip .1in 
\underbar{From Section 3.2 :} 1, 3, 4, 5, 7, 10, 11.

\vskip .1in

\underbar{From Section 3.3:} 1, 2, 3, 4, 5, 7, 12, 14, 15.
\vskip .1in 
\subhead{ Assignment 8.  Due Thursday 5/1/2012  }\endsubhead
\vskip .1in 
\underbar{From Section 3.5:} 1, 2, 3, 7, 8.
\vskip .1in 
\underbar{From Section 5.1:}  2, 7, 8, 12.
\vskip .1in 
\underbar{From Section 5.2:} 1, 2a), 6.
\vskip .2in

\underbar{Hints} \, \underbar{For $5.1 \, \#7$:} \, for each $n \ge 1$ choose an $x$ in $[0,1]$  such that $ n \,x=1$. Call that $x$, $x_n$ and compute $f_n(x_n)$.
\vskip .1in
\underbar{For $5.1\,  \# 8$:} \, for each $n \ge 1$ choose an $x$ in $[0,1]$  such that $ \dfrac{x}{n} =1$. Call that $x$, $x_n$ and compute $f_n(x_n)$.
\vskip .1in
\underbar{For $5.1 \, \#12$:} \, Given $\varepsilon >0$, write  $| f_n(x_n) - f(x_0)| \leq | f_n(x_n) - f(x_n) | + | f (x_n) - f(x_0)| $ and find $N=N(\varepsilon)$ so that  (a)  \, the first term on the r.h.s of the inequality is less than $\varepsilon/2$ thanks to the {\it uniform convergence} of $f_n$ to $f$; \, \, 
(b)  \, the second term on the r.h.s of the inequality is less than $\varepsilon/2$ thanks to the {\it continuity} of $f$
\vskip .1in
\underbar{For $5.2 \, \#2a)$:} first prove that the sequence of functions $f_n(x)= (x + \frac{1}{n})^2$ converges uniformly to the function $f(x)= x$ on $[0,1]$
 as $n$ goes to infinity. Then use Theorem 5.2.2 to compute. 
\vskip .1in
\underbar{For $5.2 \, \#6$:}  Denote by $f$ the limiting function and write $|f_n(x)| \leq |f_n(x)  - f(x)| + |f(x)|$.  

First note that since the convergence is uniform on $[0, 1]$, $f$  must be continuous (why?) and  hence bounded (why?).  
Second,  prove that there exists $N$ (think of $\varepsilon=1$) such that for all $n \geq N$ the first term on the right hand side is less than $1$.  Third, note that each of the remaining functions $f_n$, $1 \le n \le N-1$ is continuous and bounded on $[0,1]$ (and there are only a finite $N-1$, a finite number of them). 

Finally, put  all the ingredients together to conclude!


\vskip .2in



\head{\underbar{Special Projects} ({\bf  Due no later than 1PM on Friday 5/04/12} )}\endhead
\vskip .2in 
\underbar{Special Project I:}\,  Do Project $\#1$ at the end of Chapter 2 (page 68)
 \vskip .2in

\underbar{Special Project II:}\,  Do Problem 14 of Section 2.4
\vskip .2in

\underbar{Special Project III:}\,  Do Project $\#5$ at the end of
Chapter 2 (page 70)

\vskip .2in 
\underbar{Special Project IV:}\,  Do Problem 13 of Section 3.2

\vskip .2in

\underbar{Special Project V:} \, 
\vskip .05in
Let $f$  be a continuous function on $\IR$ such that the improper integral $\int_{-\infty}^{\infty} \, f(x) \, dx < \infty$.   
\vskip .05in
Let $f_n$ be a sequence of continuous functions defined on $\IR$ such that $f_n$ converge uniformly to $f$ on every finite, closed interval $[a, b]$ of $\IR$.  
\vskip .05in
Suppose that there exists a {\bf continuous} function $g: \IR \to \IR$ such that: \,   

\quad (i) \, \, $g(x) \geq 0$

\quad (ii) \, \, the improper integral $\int_{-\infty}^{\infty} \, g(x) \, dx < \infty$,   \quad

\quad (iii)  \, \,  for all $n \ge 1$ and all $ x \in \IR$ we have that  $|f_n(x)| \leq g(x)$  and also $|f(x)| \leq g(x)$.  

\vskip .2in 

\quad (a) \, Prove that each of the improper integrals  $\int_{-\infty}^{\infty} \, f_n(x) \, dx < \infty$
\vskip .1in 
%\quad b) \, Prove that $ | f(x) | \leq g(x)$ for all $x \in \IR$ (similar argument to $5.2 \, \#6$ but writing $| f(x) | \leq $| f(x) - f_n(x) |+ $| f_n(x) |
\vskip .1in 
\quad (b) \, Prove that  $$  \lim_{n \to \infty} \, \int_{-\infty}^{\infty} \, f_n(x) \, dx \, =\, \int_{-\infty}^{\infty} \, f(x) \, dx.  $$

\vskip .2in
\underbar{Hint for b)} The improper integral  $\int_{-\infty}^{\infty} \, g(x) \, dx = \lim_{M \to \infty} \int_{-M}^{M} \, g(x) \, dx $ and since 
$\int_{-\infty}^{\infty} \, g(x) \, dx <\infty$ then the limit in $M$ of the sequence $\int_{-M}^{M} \, g(x) \, dx $ of real numbers (why are each of these finite?)  exists.  

Hence given $\varepsilon>0$ there exists an $M_0=M_0(\varepsilon)>0$ such that $$ \Biggl|\int_{-\infty}^{\infty} \, g(x) \, dx - \int_{-M}^{M} \, g(x) \, dx \Biggr|\, =\,  \Biggl| \int_{ |x| > M} \, g(x) \, dx \Biggr| \, \leq \, \varepsilon \qquad  M \geq M_0. $$

To prove part for b),  you need to consider 

$$ \Biggl|\int_{-\infty}^{\infty} \, f_n(x) \, dx - \,\int_{-\infty}^{\infty} \, f(x) \, dx \Biggr|\, = \Biggl|\int_{-\infty}^{\infty} \, \Biggl( f_n(x) \,-\,  f(x) \Biggr) \, dx \Biggr| \qquad  (\dagger)$$ 

Next, rewrite the r.h.s in  ($\dagger$)  as the sum of two integrals, one over the set   ${ |x| \leq M}$  and  the other over the set ${ |x| > M}$ and use triangle inequality  to bound ($\dagger$) by 

$$\Biggl|\int_{|x| \leq M} \, \Biggl( f_n(x) \,- \, f(x) \Biggr)\, dx \Biggr| \, + \, \Biggl|\int_{|x| > M} \, \Biggl( f_n(x) \,  - \,f(x)  \Biggr) \, dx \Biggr|.$$ 

Use uniform convergence over the set ${ |x| \leq M}$. For the integral over the set,  $|x| > M$, use triangle inequality, the hypothesis (iii) and part (b) to bound each term by integrals over  $|x| > M$ of $g(x)$. 

\vskip .1in
Put all the pieces together to conclude!.


\enddocument

\vskip .2in 


\subhead{ Assignment: for you  {\bf To do} but do not to turn in}\endsubhead

\underbar{From Section 5.2:} 1, 2, 6, 9.

\vskip .1in 

\subhead{First read/review on your own the results in 6.1 \& 6.2. Then {\bf do} }\endsubhead

\vskip .1in 
\underbar {From Section 6.3} 1, 2, 6, 7, 9.  
\vskip .3in 




\underbar{{\bf Extra Problems on Limits*--- Practice for Midterm Exam. (Do not turn in)}}

\vskip .1in 
${\bold 1}$)  Prove that $\, \lim_{n \to
\infty}  \dfrac{2 n + 11}{n + 3} = 1$ using the definition of
convergence ( similar to homework 3b) \& 2.1 page 34). 

\vskip .1in 

${\bold 2}$) Prove that $\, \lim_{n \to
\infty}\sqrt{ \dfrac{5 n + 3}{n} } = \sqrt{5}$ using the definition of
convergence ( similar to Example 2 page 30). 
\vskip .1in 


${\bold 3}$) Prove that $a_n= \dfrac{5}{\sqrt{n}} + 3 $ 
is a Cauchy sequence using just the definition (similar to homework 1)
\& 2.4 page 50 ) .

\vskip .1in

${\bold 4}$) Prove that $a_n= \dfrac{4 n - 3}{n + 7} $ 
is a Cauchy sequence using just the definition (similar to Example 1 
\& 2.4 page 45) . 
                                                          
\vskip .1in







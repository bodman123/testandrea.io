\magnification=1200
\input amstex

\documentstyle{amsppt}
\NoBlackBoxes
%\NoPageNumbers
\pagewidth{16.5truecm}
\pageheight{22truecm}

{\catcode`@=11
\gdef\nologo{\let\logo@\empty}
\catcode`@=12}
\nologo
%\hcorrection{.3in}

\centerline{ \bf  M534H   HOMEWORK-- Spring 2019}
\vskip .1in
\centerline{ Prof. Andrea R. Nahmod }
\vskip .3in

\document

{\underbar{Do not turn in problems that have an  $^\ast$ next to the number.}}

\vskip .2in

$\bullet$\, {\bf{Set 1. Due date: 01/31/2019 }}
\vskip .2in



\underbar{Section 1.1}:  \quad 2, 3, 4, 10, 11, 12

\vskip .1in 
\underbar{Additional Work}:   Read carefully Appendices A1. A2 and A3.

\vskip .3in 


$\bullet$\, {\bf{Set 2. Due date: 02/14/2019 }}

\vskip .2in

\underbar{Section 1.2}:  \quad   1, 2, 3, 4, 5, 6. 

\vskip .2in 

\underbar{Additional Problem 1}: Solve the transport equation $ 5 u_x - 6 u_y =0$ together with the auxiliary condition that $u(x, 0) = 4 x^3$.

\vskip .15in

\underbar{Additional Problem 2}: Solve the inhomogeneous transport equation $2 u_x + 3 u_y = 1.$ 

\vskip .15in 

\underbar{Additional Problem 3}:  Solve the linear homogeneous equation $u_x  +  u_y  +  u =0.$

\vskip .15in

\underbar{Additional Problem 4}:   a) Check that $$u(x,y) =  \frac{1}{4} ( e^{x + 2y} - e^{x- 2y} )$$ solves the 
inhomogeneous equation $$u_x  + u_y  + u = e^{x + 2y}.$$
\smallskip

b)  Next use the additional problem 3 to write the \underbar{general form} of the solutions to $$u_x  + u_y  + u = e^{x + 2y}.$$ 
\smallskip

c)  Find the solution to $u_x  + u_y  + u = e^{x + 2y}$ that also satisfies $u(x, 0) =1$.

\vskip .2in 

\underbar{Section 1.3}: \quad 6,  9,  10 (in $\Bbb R^3$), 11$^{*}$.

\vskip .1in 

\underbar{Extra Problem (do not turn in)}:  Prove the {\it Second Vanishing Theorem} in A.1. page 416

\vskip .2in 


\underbar{Section 1.4}:  Read Section 1.4. 
\smallskip 

\underbar{Section 1.4}:  \, 1. 

\vskip .3in

$\bullet$\, {\bf{Set 3. Due date: 02/28/2019 }}

\vskip .2in

\underbar{Additional Problem 1}:   a)  Find the general solution to Problem 8 in Section 1.2. 
Specify what method you are using and explain step by step your work. Show all your work.
\smallskip

b)  Choose $a=2, b=5$ and $c=29$  and find the solution  $u(x,y)$ to part a) that also satisfies $u(x,0)= e^{-3x}$.

\smallskip

c) Check that  $\frac{1}{6}  ( e^{x+y} - e^{3x-y} ) $ is a particular solution to the inhomogeneous equation
$$  2 u_x + 5 u_y + 29 \,u =  ( 6  e^{x+y} -  5 e^{3x-y} ) $$
\smallskip

d)  Use part a) $a=2, b=5$ and $c=29$  together with part c) to find the \underbar{general solution} to the inhomogeneous equation  
$$  2 u_x + 5 u_y + 29 \, u =  (6  e^{x+y} -  5 e^{3x-y} )  $$

\vskip .1in 

\underbar{Section 1.5}:  \quad 1,  4, 5, and:
\vskip .1in 

%
%\underbar{Problem 5 of Section 1.5}:  Consider the equation $$u_x + y u_y =0$$ 
%
%with auxiliary condition $u(x,0)= \phi(x)$.
%
%(a) \, For $\phi(x)= x$, for all $x$, show that no solution exists.
%
%(b) \, For $\phi(x)=1$ for all $x$, show that there are many solutions. 
%
%\vskip .1in

\underbar{Problem 6 (modified)}:  Solve the equation $u_x \, +\, 2 x \, y^2\, u_y \,=\,0$ and find a solution that satisfies the auxiliary condition $u(0, y) =  y$.  

\vskip .1in 

\underbar{Section 1.6}:  \quad 1, 2, 4

\vskip .1in 

\underbar{Additional Problem 2}:  \, Find the regions in $\Bbb R^2$  where $x^2\, u_{xx} + 4 u_{xy} +  y^2 \, u_{yy} \, =\,0$ 
is respectively elliptic, parabolic, hyperbolic. Plot these regions. 

\vskip .1in 

\underbar{Section 2.1}: \quad 1, 2, 8.  
\medskip
\underbar{Hint for 8}:  Recall  a function $f$ on $\Bbb R$ is odd if $f(x)= - f(-x)$.

\vskip .3in

$\bullet$\, {\bf{Set  4. Due date: 03/07/2019 }}

\vskip .1in 

\underbar{Additional Problem 1}:  First find the solution to the linear homogeneous wave equation with wave speed $1$ and 
with initial conditions $u(x, 0) = \sin x, \, u_t(x, 0)= 0$. \, Then calculate  $u_t(0, t)$. 

\vskip .2in
\underbar{Additional Problem 2}  \,\, Find the solution to the wave equation $ u_{tt} - 4 u_{xx} =0$ and with initial conditions $u(x, 0) = \sin x, \, u_t(x, 0)= 10 $. Calculate then $u_t(0, t)$. 

\vskip .2in

\underbar{Section 2.1}:  \quad 9,  10, 11 

\smallskip
\underbar{Hint for 11}:   \, $ -\frac{1}{16} sin(x+t) $ is a particular solution. Check it!.

\bigskip


\underbar{Additional Problem 3}:  Find the general solution to \, $ u_{xx} + u_{xt} -  10 u_{tt} \,=\, 0$  (check whether is hyperbolic first).
\vskip .1in

\underbar{Additional Problem 4}:  Find the general solution to \, $ u_{xx} + 2 u_{xt} -  20 u_{tt} \,=\, 0$  (check whether is hyperbolic first).

\vskip .2in

\underbar {\it Bonus Problem*:}  \, Read Example 2 in Strauss' book pages 36-37. Then do Problem 5  Section 2.1  (do not turn in). 




\vskip .3in

$\bullet$\, {\bf{Set  5. Due date: 03/21/2019 }}

\vskip .1in
\underbar{Additional Problem 1}: Find the solution the IVP  
$$\cases   & u_{xx} - 6 u_{xt} +  5 u_{tt} = 0, \qquad x \in \Bbb R, \, t>0  \\
&  u(x, 0) = x^2  \\
&  u_t(x, 0) =0 
\endcases $$
Check first whether the second order PDE is hyperbolic. 

\vskip .1in 

\underbar{Additional Problem 2}  \, Consider the wave equation in $1$D with {\it damping} 
$$ u_{tt} = c^2\, u_{xx} - k u - r u_{t} \, \qquad k, r >0$$  show that the {\it energy functional} $$E(t) \, =\, \frac{1}{2} \int_{-\infty}^{\infty} \, |u_t|^2  + c^2 |u_x|^2 + k |u|^2 \, dx $$ satisfies $dE/dt \le 0$; that is {\it energy decreases}. Assume $u$ and its derivatives vanish as  $x \to \pm \infty$. 

\vskip .2in

\underbar{Section 2.2}: \quad 1, 2, 3 
\vskip .1in

\underbar{Additional Problem 3:}  Let $u\,= \,u(\bold{x}, t)$ be a solution to the wave equation $u_{tt} - \Delta u =0$ in $\Bbb R^2$.  Assuming that $ \nabla u \to 0$ fast enough as $| \bold{x}| \to \infty$ prove that $$ E(t) \, =\, {\int}_{\Bbb R^2} \, \,  |u_t|^2 \, + \,  |\nabla\, u |^2 \, \, dx dy $$  is constant in $t$ for all time $t$. 

\vskip .3in 



$\bullet$\, {\bf{Set  6. Due date: 03/28/2019 }}
\vskip .1in

\underbar{Section 2.3}:\quad 2, 4, 5, 6, 7a). 
\vskip .1in 
\underbar{Hint for 4b)}:  Do not solve explicitly. Rather prove that $u(1-x, t)$ also solves the equation and then apply the uniqueness theorem.

\vskip .2in
 \underbar{Section 2.4}: \quad  3, 5 (do only (a)(b)(c)), 8.

\vskip .3in

$\bullet$\, {\bf{Set 7. Due date: \underbar{Tuesday} 04/09/2019  }}
\vskip .1in
 \underbar{Section 2.4}: \quad   2,  4,  9,  11a)b), 15, 16.
 
\vskip .1in
{\underbar{Additional Problem 1} a) Show that the function $ u(x, t) = e^{-k t} \sin(x) $ solves the heat equation $ u_t - k u_{xx} = 0$.
\medskip
b) Find a relationship between the constants $a$ and $b$ so that $u(x,t)= e^{-a t} \cos(b x) $ is a solution to $ u_t - k u_{xx} = 0$ (assume $\cos(bx) \neq 0$). 

\vskip .1in
%\underbar{Additional Problem 2}:   a) Show that the fundamental solution on $\Bbb R$ given by,  $$\Gamma_k(x,t):= \frac{1}{\sqrt{4 \pi kt}}  e^{\frac{-x^2}{4kt}}$$ solves the heat equation $ u_t - k u_{xx} = 0$.
%
%
%\vskip .1in
%
%b) Show that $$ \int_{-\infty}^{\infty} \, \Gamma_k(x,t) \, dx \, =\, 1 \qquad \text{ for all} \quad t >0.$$
%\underbar{Hint} Change variables $y:= \frac{x}{\sqrt{4kt}}$ and use that $\int_{-\infty}^\infty  e^{-y^2}  \, dy = \sqrt{\pi}$
%\medskip
%c) Use b) and another simple change of variables to show that $$ \int_{-\infty}^{\infty} \, \Gamma_k(x-z ,t) \, dz \, =\, 1 \qquad \text{ for all} \quad t >0.$$
%
%\vskip .1in
%{\underbar{Additional Problem 2}: Find the solution to the following Cauchy IVP for the heat equation:
%$$\cases u_{t} \,- \,  k u_{xx} \, =\, 0,   \qquad  x \in \Bbb R, \, \, \, t>0  \\
%  u(x, 0) \, =\, 2 , \quad x \in \Bbb R \endcases, $$ 
%  by using additional problem 3c) and that $u(x,t) = (\Gamma_k (\cdot, t) \ast g)(x)$ where $g$ is the initial datum. Check your answer indeed solves the Cauchy IVP above.
%   
 
 
\vskip .1in
\underbar{Additional Problem 2}:  This problem pertains the heat equation on the whole real line. Recall Gauss' {\it error function} 
$$ \text{erf}(x) \, :=  \, \frac{2}{\sqrt{\pi}}\, \int_0^x  e^{-z^2} dz .$$ 
a) Prove that all solutions to the heat equation 
$u_t - k u_{xx} =0$ of the form  $ u(x, t) = v(\frac{x}{\sqrt{t}})$, for $x \in \Bbb R$ and $t >0$ have the form $$(\dagger) \qquad \quad  u(x, t) = C_1 + C_2 \, \text{erf}(\frac{x}{\sqrt{4k t}}). $$
b) Prove that by choosing $C_2$ suitably in $(\dagger)$ the fundamental solution $\Gamma_k(x,t) = u_x(x, t)$.
\medskip 
\underbar{Hint.} For a):   First change variables $y:= \frac{x}{\sqrt{t}}$, whence $$\frac{\partial y}{\partial t}=  -\frac{x}{2 t \sqrt{t}}, \qquad
\frac{\partial y}{\partial x}=  \frac{1}{\sqrt{t}}, \qquad \text{and} \quad \frac{\partial^2 y}{\partial x^2}= 0. $$ Then prove that if $u(x,t)$ is a solution as above then $v(y)$ solves the ODE: 
$$ \frac{y}{2k} v^{\prime}(y) \, +\, v^{\prime\prime}(y) \, =\, 0, \, \, \, {\text{which can be viewed as a {\it first order } ODE for}} \quad w= v^{\prime}(y).$$ Solve then first $ \frac{y}{2k} w(y) \, +\, w^{\prime}(y) \, =\, 0$ and then by further integration find $v(y)$ expressed in terms of Gauss' {\it error function}. Recall
finally that $u(x, t) = v(\frac{x}{\sqrt{t}}).$

\vskip .1in

{\underbar{Additional Problem 3}:  Consider the Cauchy IVP for the heat equation:
$$(\dagger\dagger) \qquad \cases u_{t} \,- \,  k u_{xx} \, =\, 0,   \qquad  x \in \Bbb R, \, \, \, t>0  \\
  u(x, 0) \, =\, H(x) , \quad x \in \Bbb R \endcases $$
where $H(x)$ is the Heaviside function, $H(x) = 0$ if $x <0$ and $H(x) =1$ for $x \geq 0$.  
\smallskip

Using the general formula given the additional problem 3a) above,  in find the constants $C_1$ and $C_2$ so that $u(x,t)$  \underbar{also} satisfies the initial condition and hence is a solution to $(\dagger\dagger)$. 

In other words find $C_1$ and $C_2$ so that  for $x<0$ 
$$ 0 = u(x, 0^{+}) = \lim_{t \to 0^{+}}  C_1 + C_2 \, \text{erf}(\frac{x}{\sqrt{4k t}}) $$ 
and for $x  >0$, 
$$ 1 = u(x, 0^{+}) = \lim_{t \to 0^{+}}  C_1 + C_2 \, \text{erf}(\frac{x}{\sqrt{4k t}}) $$
Note in the first case get an integral between $0$ and $-\infty$ while on the second you get an integral between $0$ and $\infty$.

After finding $C_1$ and $C_2$ write down the final expression for $u(x,t)$ solving $(\dagger\dagger)$ in terms of the \, $ \text{erf}$ \,function and the explicit constants $C_1$ and $C_2$.


\vskip .15in

\underbar{Section 2.5*}:\quad 1$^{*}$  

\underbar{Hint} Consider the example in homework 5 section 2.1 (the hammer blow) and note that 
since there is no boundary, if there was a maximum principle it would assert that it has to be attained initially. But ....

\vskip .3in 



$\bullet$\, {\bf{Set 8. Due date: 04/23/2019  }}


\vskip .15in

\underbar{Section 4.1}:\quad  2, 3.

\smallskip

\underbar{Hint} In 3) proceed as in the heat equation and  keep the complex $i$ next to $T(t)$.  Recall that the
solution to the  ODE    $T'(t) = i \lambda T(t)$ is  $T(t) = A e^{- i \lambda t} $.

\vskip .2in 

\underbar{Additional Problems (section 4.1)} Separate variables to solve the following problems. 
\vskip .1in 

A1.  \quad  $u_{tt} - u_{xx} =0$ in $0<x<3$ with boundary conditions $u(0, t)= u(3, t)=0$

\vskip .1in 

A2. \quad \, $u_t \,=\, u_{xx} $ in $0< x< \pi$ with boundary conditions $u(0,t)\, = \,u(\pi, t) \,=\,0$

\vskip .1in 

A3.  Let $g$ be a smooth function on $[0,1]$, \, $g(0)=g(1) =0$.  Consider the Dirichlet BIVP:
\vskip  .1in
$u_t -  u_{xx} + 3 u= 0\, $ in $0< x< 1$  with $u(0,t)\, = \,u(1, t) \,=\,0$ and  $u(x, 0)= g(x)$  \quad  $(\oplus)$
\vskip  .1in
a)  Consider the change of variables $u(x,t) = e^{-3 t} v (x, t)$ and prove that $v$ solves 
\vskip  .1in
$v_t -  v_{xx}= 0$ \, in $0< x< 1$  with $v(0,t)\, = \,v(1, t) \,=\,0$ and $v(x, 0)= g(x)$.   \quad  $(\dagger)$
\vskip  .1in
b)  Use the method of separation of variables to find  $v$,  the solution to $(\dagger)$.
\vskip .1in
c)  Use a) and b) to find $u$, the solution to  $(\oplus)$.


\vskip .2in

\underbar{Section 4.2}:
%\qquad \, 1,  3
\smallskip
\underbar{First:}  Read first the last part of section 4.2 (page 91) where there is an example with mixed boundary conditions. 

\vskip .2in

\underbar{Additional Problem 4} Separate variables to solve the following problem for the heat equation with mixed boundary conditions:
$u_t - 5 u_{xx} \,=\, 0$ in $0<x< L$ with boundary conditions $u_x(0,t) = u (L,t)=0$

\vskip .3in


\underbar{Section 5.1} \, \, \,  9 
\vskip .1in 
\underbar{Hint for 9:}  Use the trig. identities $\cos (2x) = \cos^2(x) - \sin^2(x) =  2 \cos^2(x) - 1$  to re-express the initial velocity and immediately obtain its cosine "expansion". 

%%
%%   saque de la 5.1  el problema 6
%%

\vskip .1in


\underbar{Section 5.2} \, \, \,  Read the definitions of even and odd functions in page 114 ( formulas (3) and (4)).    Then do the following problems:   \quad    3,  \, 10




%%   saque problemas 7, 9, 11.

\vskip .15in 

\underbar{Additional Problem 5} \,  Let $\phi : \Bbb R \to \Bbb R$ be a periodic function with period $p$; that is $\phi(x + p) = \phi(x), \, \, \forall x \in \Bbb R$. Assume that 
$\phi$ is integrable on any finite interval. 
\smallskip

(a) Prove  that for any $a, b \in \Bbb R$ 
$$  \int_a^b \phi(x) d x  = \int_{a+ p}^{b+p} \phi(x) dx = \int_{a- p}^{b - p} \phi(x) dx $$
\smallskip

(b) Prove  that for any $a \in \Bbb R$ $$  \int_{-p/2}^{p/2} \phi(x+a) d x  = \int_{- p/2 +a }^{p/2+a} \phi(x) dx = \int_{-p/2}^{p/2} \phi(x) dx $$

\smallskip 

Note then  that for any $a \in \Bbb R, \quad  \int_{-p/2}^{p/2} \phi(x+a) d x = \int_{-p/2}^{p/2} \phi(x) dx$ and thus in particular  that $\int_a^{a+p}  \phi(x) \, dx$ does not depend on $a$, as we discussed in class (section 5.2, Strauss). 

\vskip .3in 

$\bullet$\, {\bf{Set 9. Due date: 04/25/2019  }}


\vskip .15in

\underbar{Section 5.1} \, \, \,  2a),  5

\vskip .15in

\underbar{Section 5.2} \, \, \,  4,

\underbar{Section 5.3} \, \, \, 2a)b), $3^{\dagger}$ 
\vskip .1in 
Problem $3^{\dagger}$ means that you should do this problem for zero Neumann boundary conditions instead of the mixed ones.
That is consider instead  the given wave equations with $u_x(0, t)=0= u_x(\ell, t)$ and the same initial data $u(x, 0)= x$ and 
$u_t(x, 0)=0$.


%%%  saque de la 5.3 el problema 1, 4a)b), 5a)  y el 10  

\vskip .2in 

\underbar{Section 5.4} \, \, \,  5, 6, 8a)
%%%  saque de la 5.4  los problemas 1 y  7, 



\vskip .15in 



\newpage



\centerline{\bf Special Assignments}
\medskip

Please do the following but do not turn in yet.  The solutions to these special projects must be typed using Latex. Due date: TBA
\medskip


\underbar{\bf Special Project 1}
\medskip

Consider the {\it initial value problem for the wave equation} on $\Bbb R$:

$$\cases  u_{tt} - u_{xx} \,=\,0  \\
u(x, 0) \,=\, \phi(x) \\
u_t(x, 0) \,=\, \psi(x) \endcases $$ where $\phi, \psi :\Bbb R \to \Bbb R$ are two smooth given functions (data).  Let $x_0 \in \Bbb R$ $t_0 >0$ be fixed and \underbar{suppose} that $\phi(x)$ and $\psi(x)$ vanish for all 
$x$ in the interval $[x_0 - t_0, x_0 + t_0]$. 

\vskip .2in


\underbar{\it Finite Propagation Speed Theorem:}.   The solution  $u(x,t)$ to the initial value problem above vanishes for all  $(x,t)$ within  $\Cal C$, the domain of dependence of $(x_0, t_0)$.  Recall $$\Cal C :=  \{ (x,t)  :  0 \le t \le t_0  \, \text{ and }  \, x_0 -(t_0 - t)  \le x \le x_0 + (t_0 -t) \}.$$

\vskip .2in 


{\bf Remark:} The Theorem is also valid in higher dimensions but for simplicity I will ask you to prove it only in one (space) dimension. In one dimension, one can trivially prove the above theorem directly using the representation formulas for the solution $u(x,t)$  in terms of the initial data which are available in one dimension.  Or, one could prove it \underbar{without} using this explicit representation of $u$, but by using the \underbar{{\it energy method}} instead --as we have seen in class-. This proof is a bit harder but the advantage of the method is that it also works in higher dimensions.

\vskip .1in 

{\bf The project consists then to prove the Finite Propagation Speed Theorem above using the \underbar{energy method}. }
\vskip .2in 
To do so, for each \, \,  \,$0 \le t \le t_0$, \,  let  \,\, \, $I_t : = [x_0-(t_0-t), x_0 + (t_0-t) ]$. 

Note $I_t$ is contained in the interval $(x_0 - t_0, x_0 + t_0)$. Define the modified energy:

$${\widetilde E} (t) = \frac{1}{2} \int_{I_t} \, |u_t|^2 + |u_x|^2 \, dx $$

Note ${\widetilde E} (t) \ge 0$ for any $t$ and that  $\Cal C = \bigcup_{ 0 \le t \le t_0}  I_t$.
The goal is to show that for each $0 \le t \le t_0$, $u(x,t)=0$ for all $x \in I_t$. Do so by proving the following: 

\vskip .1in


(1) \, Prove that ${\widetilde E} (t)$ is a decreasing function of $t$ by showing that 
$\dfrac{d{\widetilde E}}{dt} \le 0$ 

\vskip .1in

To compute the derivative in time \underbar{use}: (see A.3 Theorem 3 in Strauss's book p.421).

$$\frac{d}{dt} \int_{a(t)}^{b(t)} \, F(x,t) \, dx \,= \,  \int_{a(t)}^{b(t)} \, \frac{d}{dt} F(x,t) \, dx  +  [ F(b(t), t) b^{\prime}(t) - F(a(t), t) a^{\prime}(t) ] $$

\vskip .2in

(2) Show that ${\widetilde E} (0) =0$  
 
 \vskip .2in 
 
(3) By (1) you then have that $\widetilde E(t) \le \widetilde E (0)$ for any $0 \le t \le t_0$ and 
by (1) you can conclude that ${\widetilde E} (t)  =0$ for any $0 \le t \le t_0$.
\underbar{Prove} then that this implies that $u(x,t)=0$ for  any $x \in I_t$ and any $0 \le t \le t_0$.

\vskip .3in

\underbar{\bf Special Project 2}
 \medskip

Consider now the initial value problem for the diffusion equation on the {\bf whole} real line $\Bbb R$ with $k=1$:
 $$ {(\ast)} \qquad \qquad  \cases u_{t} \,- \,    u_{xx} \, =\, 0,   \qquad  x \in \Bbb R, \, \, \, t>0 \\
  u(x, 0) \, =\, e^{2x}, \quad   x \in \Bbb R \\
 \endcases $$ 
Use the fact that the solution $u(x,t)$ is obtain by the convolution of the fundamental solution with the initial data; that is by:
$$ u(x, t) \, = \, \int_{-\infty}^\infty \, \Gamma_k ( x-y,  t) \, e^{2y} \, dy $$
to find the function that $u(x,t)$ equals to. Check that your answer solves indeed  $(\ast)$. 

\medskip

{\underbar {Follow the following Hints:}}
\medskip

 1)  Recall that in 1D,  $\Gamma_k(x-y, t) = \frac{1}{\sqrt{4 \pi k t}} e^{-\frac{(x-y)^2}{4kt}}, \, \,  t>0$  (in Strauss notation this is $S(x-y, t)$).   Note that here we have $k=1$. 

\smallskip

2)  After developing the square in $\Gamma$, collect all the exponents of the exponentials and 
 {\bf complete the square in the $y$} variable. Note that terms that have only $x$ and $t$ in the exponents can 
 come out of the integral. 
 \smallskip
 3) You may use that $\int_{-\infty}^{\infty} \, e^{-p^2} \, dp = \sqrt{\pi}$. 
 You may find  the change of variables $p= \frac{y - (x +4t)}{\sqrt 4t}$ useful. 

\vskip .3in

\underbar{\bf Special Project 3}
 \medskip
 
a) \,(wave with a source) Find the solution to the following inhomogeneous wave equation on $\Bbb R$. Evaluate all the integrals to obtain a nice formula for the solution

$$u_{tt} - 9 u_{xx} \, =\, x \, t  \qquad u(x, 0)\, =\, \sin(x) \qquad u_t(x, 0)\,=\, 1 + x $$

\vskip  .3in 

b) \, (wave on the half line). Find the solution to the following wave equation on the half-line {\underbar{using the reflection method.} 
Show all your work.

$$\cases u_{tt} - 4 u_{xx} \, = \, 0 \\  
u(x, 0) = 1, \quad  u_t(x, 0) = 0 \\
u(0, t) = 0  \endcases $$

The solution has a jump discontinuity in the $(x, t)$ plane. Find its location (explain).


\vskip .3in

\underbar{\bf Special Project 4}
 \medskip
a)  Find the Fourier \underbar{cosine} series of $\phi(x) = x^2$ \,\, for  \,\, $x \in [0, 1]$
\medskip
b) State in what sense does the cosine series in part a) converges to the function $x^2$ on  [$[0,1]$. 
\medskip
c)   Use separation of variables and the superposition principle to find the general solution to the following 
boundary value problem for the heat equation on an interval:
 $$ {(H)}  \qquad \qquad  \cases u_{t} \,- \,    u_{xx} \, =\, 0,   \qquad  0< x < 1,  \, \, \, t>0 \\
  u_x(0, t) \, =\, 0  \, =\,  u_x(1, t)\quad  t>0   \\
 \endcases $$ 
 In the course of your proof do an analysis of all the possible eigenvalues ($\lambda >0$, $\lambda =0$, $\lambda <0$ ) to the problem,
 $$ \cases X^{\prime\prime} + \lambda X(x) = 0   \qquad  0< x < 1 \\
  X^{\prime}(0) \, =\, 0  \, =\,  X^{\prime}(1) 
 \endcases $$ 
\medskip 
 d) Find the particular solution to (H) that also satisfies the initial condition that $u(x, 0) = x^2$, \, for \, $0 < x < 1$.
\enddocument




$\bullet$\, {\bf{Set 9. Due date: 04/25/2018}}
\vskip .2in 

\underbar{Section 4.2}:\quad 1,  3
\smallskip
\underbar{Hint.}  For 1) read first the last part of section 4.2 (page 91) where there is an example with mixed boundary conditions. 

\vskip .15in

\underbar{Additional Problem 1} Separate variables to solve the following problem for the heat equation with mixed boundary conditions:
$u_t - 5 u_{xx} \,=\, 0$ in $0<x< L$ with boundary conditions $u_x(0,t) = u (L,t)=0$

\bigskip

\underbar{Section 5.1} \, \, \,  2a),  5, 9 
\vskip .1in 
\underbar{Hint for 9:}  Use the trig. identities $\cos (2x) = \cos^2(x) - \sin^2(x) =  2 \cos^2(x) + 1$  to re-express the initial velocity and immediately obtain its cosine "expansion". 

%%
%%   saque de la 5.1  el problema 6
%%

\vskip .1in

\underbar{Section 5.2} \, \, \,  3, 4, 10, 11 

%%   saque el 7 y el 9

\vskip .1in 

\underbar{Additional Problem 2} \,  Let $\phi : \Bbb R \to \Bbb R$ be a periodic function with period $p$; that is $\phi(x + p) = \phi(x), \, \, \forall x \in \Bbb R$. Assume that 
$\phi$ is integrable on any finite interval. 
\smallskip

(a) Prove  that for any $a, b \in \Bbb R$ 
$$  \int_a^b \phi(x) d x  = \int_{a+ p}^{b+p} \phi(x) dx = \int_{a- p}^{b - p} \phi(x) dx $$
\smallskip

(b) Prove  that for any $a \in \Bbb R$ $$  \int_{-p/2}^{p/2} \phi(x+a) d x  = \int_{- p/2 +a }^{p/2+a} \phi(x) dx = \int_{-p/2}^{p/2} \phi(x) dx $$

\smallskip 

Note then  that for any $a \in \Bbb R, \quad  \int_{-p/2}^{p/2} \phi(x+a) d x = \int_{-p/2}^{p/2} \phi(x) dx$ and thus in particular  that $\int_a^{a+p}  \phi(x) \, dx$ does not depend on $a$, as we discussed in class (section 5.2, Strauss). 




\vskip .15in 

$\bullet$\, {\bf{Set 10. Due date: 05/01/2018  }}

\vskip .1in 

\underbar{Section 5.3} \, \, \, 2a)b), $3^{\ast}$, 6  
\vskip .1in 
$^{\ast}$ in problem 3 means that you should do this problem for zero Neumann boundary conditions instead of the mixed ones.
That is consider instead  the given wave equations with $u_x(0, t)=0= u_x(\ell, t)$ and the same initial data $u(x, 0)= x$ and 
$u_t(x, 0)=0$.


%%%  saque de la 5.3 el problema 1, 4a)b), 5a)  y el 10  

\vskip .1in 

\underbar{Section 5.4} \, \, \,  5, 6, 8a)
%%%  saque de la 5.4  los problemas 1 y  7, 
\newpage


\centerline{\bf Special Assignments}
\medskip

Please do the following but do not turn in yet.  Due date:  TBA.
\medskip

\underbar{\bf Special Project 1}
\medskip

Consider the {\it initial value problem for the wave equation} on $\Bbb R$:

$$\cases  u_{tt} - u_{xx} \,=\,0  \\
u(x, 0) \,=\, \phi(x) \\
u_t(x, 0) \,=\, \psi(x) \endcases $$ where $\phi, \psi :\Bbb R \to \Bbb R$ are two smooth given functions (data).  Let $x_0 \in \Bbb R$ $t_0 >0$ be fixed and \underbar{suppose} that $\phi(x)$ and $\psi(x)$ vanish for all 
$x$ in the interval $[x_0 - t_0, x_0 + t_0]$. 

\vskip .1in

\underbar{Prove} the  {\it Finite Propagation Speed Theorem}. That is prove that $u(x,t)$ vanishes for all  $(x,t)$ within  $\Cal C$, the domain of dependence of $(x_0, t_0)$.  Recall $$\Cal C :=  \{ (x,t)  :  0 \le t \le t_0  \, \text{ and }  \, x_0 -(t_0 - t)  \le x \le x_0 + (t_0 -t) \}.$$

\vskip .1in 


{\bf Note:} The Theorem is also valid in higher dimensions but for simplicity prove it only in one (space) dimension. In one dimension, one can trivially prove the above theorem directly using the representation formulas for the solution $u(x,t)$  in terms of the initial data which are available in one dimension.  Or, one could prove it \underbar{without} using this explicit representation of $u$, but by using the \underbar{{\it energy method}} instead --as we have seen in class-. This is a harder proof but the advantage of the method is that it also works in higher dimensions.

This assignment is then to prove the Finite Propagation Speed Theorem using the {\it energy method}. 
To do so, for each \, \,  \,$0 \le t \le t_0$, \,  let  \,\, \, $I_t : = [x_0-(t_0-t), x_0 + (t_0-t) ]$. 

Note $I_t$ is contained in the interval $(x_0 - t_0, x_0 + t_0)$. Define the modified energy:

$${\widetilde E} (t) = \frac{1}{2} \int_{I_t} \, |u_t|^2 + |u_x|^2 \, dx $$

Note ${\widetilde E} (t) \ge 0$ for any $t$ and that  $\Cal C = \bigcup_{ 0 \le t \le t_0}  I_t$.
The goal is to show that for each $0 \le t \le t_0$, $u(x,t)=0$ for all $x \in I_t$. Do so by proving the following: 



(1) \, Prove that ${\widetilde E} (t)$ is a decreasing function of $t$ by showing that 
$\dfrac{d{\widetilde E}}{dt} \le 0$ 

To compute the derivative in time \underbar{use}: (see A.3 Theorem 3 in Strauss's book p.421).

$$\frac{d}{dt} \int_{a(t)}^{b(t)} \, F(x,t) \, dx \,= \,  \int_{a(t)}^{b(t)} \, \frac{d}{dt} F(x,t) \, dx  +  [ F(b(t), t) b^{\prime}(t) - F(a(t), t) a^{\prime}(t) ] $$

\vskip .1in

(2) Show that ${\widetilde E} (0) =0$  
 
 \vskip .1in 
 
(3) By (1) you then have that $\widetilde E(t) \le \widetilde E (0)$ for any $0 \le t \le t_0$ and 
by (1) you can conclude that ${\widetilde E} (t)  =0$ for any $0 \le t \le t_0$.
\underbar{Prove} then that this implies that $u(x,t)=0$ for  any $x \in I_t$ and any $0 \le t \le t_0$.

\vskip .3in

\underbar{\bf Special Project 2}
 \medskip
 
a) \,(wave with a source) Find the solution to the following inhomogeneous wave equation on $\Bbb R$. Evaluate all the integrals to obtain a nice formula for the solution

$$u_{tt} - 9 u_{xx} \, =\, x \, t  \qquad u(x, 0)\, =\, \sin(x) \qquad u_t(x, 0)\,=\, 1 + x $$

\vskip  .3in 

b) \, (wave on the half line). Find the solution to the following wave equation on the half-line {\underbar{using the reflection method.} 
Show all your work.

$$\cases u_{tt} - 4 u_{xx} \, = \, 0 \\  
u(x, 0) = 1, \quad  u_t(x, 0) = 0 \\
u(0, t) = 0  \endcases $$

The solution has a jump discontinuity in the $(x, t)$ plane. Find its location (explain).

\vskip .3in

\underbar{\bf Special Project 3}
 \medskip

Consider now the initial value problem for the diffusion equation on the {\bf whole} real line $\Bbb R$ with $k=1$:
 $$ {(\ast)} \qquad \qquad  \cases u_{t} \,- \,    u_{xx} \, =\, 0,   \qquad  x \in \Bbb R, \, \, \, t>0 \\
  u(x, 0) \, =\, e^{2x}, \quad   x \in \Bbb R \\
 \endcases $$ 
Use the fact that the solution $u(x,t)$ is obtain by the convolution of the fundamental solution with the initial data; that is by:
$$ u(x, t) \, = \, \int_{-\infty}^\infty \, \Gamma_k ( x-y,  t) \, e^{2y} \, dy $$
to explicitly find the function that $u(x,t)$ equals to. Check that your answer solves indeed  $(\ast)$. 

 \underbar{Hints.} 
 Recall that in 1D,  $\Gamma_k(x-y, t) = \frac{1}{\sqrt{4 \pi k t}} e^{\frac{(x-y)^2}{4kt}}, \, \,  t>0$  (in Strauss notation this is $S(x-y, t)$).  
 Note that here we have $k=1$. 

 After developing the square in $\Gamma$, collect all the exponents of the exponentials and 
 {\bf complete the square in the $y$} variable. Note that terms that have only $x$ and $t$ in the exponents can 
 come out of the integral. You may use that $\int_{-\infty}^{\infty} \, e^{-p^2} \, dp = \sqrt{\pi}$. 
 You may find  the change of variables $p= \frac{y - (x +4t)}{\sqrt 4t}$ useful. 

\vskip .3in

\underbar{\bf Special Project 4}
 \medskip
a)  Find the Fourier \underbar{cosine} series of $\phi(x) = x^2$ \,\, for  \,\, $x \in [0, 1]$
\medskip
b) State in what sense does the cosine series in part a) converges to the function $x^2$ on  [$[0,1]$. 
\medskip
c)   Use separation of variables and the superposition principle to find the general solution to the following 
boundary value problem for the heat equation on an interval:
 $$ {(H)}  \qquad \qquad  \cases u_{t} \,- \,    u_{xx} \, =\, 0,   \qquad  0< x < 1,  \, \, \, t>0 \\
  u_x(0, t) \, =\, 0  \, =\,  u_x(1, t)\quad  t>0   \\
 \endcases $$ 
 In the course of your proof do an analysis of all the possible eigenvalues ($\lambda >0$, $\lambda =0$, $\lambda <0$ ) to the problem,
 $$ \cases X^{\prime\prime} + \lambda X(x) = 0   \qquad  0< x < 1 \\
  X^{\prime}(0) \, =\, 0  \, =\,  X^{\prime}(1) 
 \endcases $$ 
\medskip 
 d) Find the particular solution to (H) that also satisfies the initial condition that $u(x, 0) = x^2$, \, for \, $0 < x < 1$.
\enddocument



\underbar{Special Project 4} 

a)  \, Separate variables to solve the following problem for the wave equation with Neumann boundary conditions:
$u_{tt} - 9 u_{xx} \,=\, 0$ in $0<x< 2$ with boundary conditions $u_x(0,t)= u_x (2,t)=0$

\vskip .2in


\enddocument







                                               



$\bullet$\, {\bf{Set 7. Due date:  Tuesday May 6  }}
\vskip .1in 
\underbar{Section 5.2} \, \, \,  3, 4, 7, 9, 10, 11, 17.

\vskip .1in 
\underbar{Section 5.3} \, \, \, 1, 2a)b), 3, 4a)b), 5a), 6, 10.

\vskip .1in 

\underbar{Section 5.4} \, \, \,  1, 5, 6, 7 , 8a)



\vskip .2in 

$\bullet$\, {\bf{Set 6. Due date:  Thursday April 24 }}
\vskip .1in 
\underbar{Section 4.3} \, \, \,   2, 4, 6, 11

\vskip .1in 
\underbar{Section 5.1} \, \, \, 3, 5, 6, 7, 9. 
\vskip .1in 

\underbar{Additional Problems for 4.3 and 5.1} 

\vskip .2in 

$\bullet$\, {\bf{Set 5. Due date:  Thursday April 10 }}

\vskip .1in

\underbar{Section 2.4}:\quad  1, 2, 7 (here you can use that the integral is $\sqrt{\pi}$ which we proved in class) 

\hskip 0.8in 8, 14, 15 16, 18

\vskip .1in 

\underbar{Section 2.5}:\quad 1  

\vskip .1in 

\underbar{Section 4.1}:\quad  2, 3
\vskip .1in 
\underbar{Additional Problems} Separate variables to solve the following problems. 
\vskip .1in 

A1.  \quad  $u_{tt} - u_{xx} =0$ in $0<x<3$ with boundary conditions $u(0, t)= u(3, t)=0$

\vskip .1in
A2. \quad $u_t=u_{xx}$ in $0<x< L$ with boundary conditions $u_x(0,t)= u (L,t)=0$

\vskip .1in

A3. \quad \, $u_t \,=\, u_{xx}\,+\, u$ in $0< x< \pi$ with boundary conditions $u(0,t)\, = \,u(\pi, t) \,=\,0$
\vskip .2in 



\underbar{Section 4.2}:\quad  1, 3

\vskip .1in





$\bullet$\, {\bf{Set 4. Due date:  Thursday March 13th }}

\vskip .1in


\underbar{Section 2.3}:\quad 2, 4, 5, 6, 7. 

\vskip .2in 
$\bullet$\, {\bf{Set 3. Due date:  Thursday March 6th.\, \underbar{Postponed to March 11th}}}

\vskip .1in

\underbar{Section 2.1}:\quad 1, 2, 5,  8, 9,  11 (Hint.   \, $ -\frac{1}{16} sin(x+t) $ is a particular solution. Check it!). 

\vskip .1in 

\underbar{Section 2.2}:  1, 2, 3, 5* 
\vskip .1in
{\bf Problem 5*}  \, Consider the wave equation in $1$D with {\it damping} 
$$ u_{tt} = c^2\, u_{xx} - k u - r u_{t} \, \qquad k, r >0$$  show that the {\it energy functional} $$E(t) \, =\, \frac{1}{2} \int_{-\infty}^{\infty} \, |u_t|^2  + c^2 |u_x|^2 + k |u|^2 \, dx $$ satisfies $dE/dt \le 0$; that is {\it energy decreases}. Assume $u$ and its derivatives vanish as  $x \to \pm \infty$. 

\vskip .1in

{\bf Additional Problems}.
\vskip .1in 
{\bf Problem A.1.} \,\, Find the solution to wave equation with initial conditions $u(x, 0) = \sin x, \, u_t(x, 0)= 0$. Calculate then $u_t(0, t)$. 

\vskip .1in

{\bf Problem A.2.}\, \, Solve the PDE \quad $ u_{xx} + u_{xt} -  10 u_{tt} \,=\, 0$



\vskip .2in


$\bullet$\, {\bf{Set 2. Due date:  Thursday February 28th }}
\vskip .2in

\underbar{Section 1.4}:  \quad  1

\vskip .1in 

\underbar{Section 1.5}:  \quad 1,  4,  5,  6 (modified; see below)

\vskip .2in 

\underbar{Problem 5 states}:  Consider the equation $$u_x + y u_y =0$$ 

with boundary condition $u(x,0)= \phi(x)$.

(a) \, For $\phi(x)= x$, for all $x$, show that no solution exists.

(b) \, For $\phi(x)=1$ for all $x$, show that there are many solutions. 

\vskip .1in

\underbar{Problem 6 states}:  Solve the equation $u_x \, +\, 2 x \, y^2\, u_y \,=\,0$ and find a solution that satisfies the auxiliary condition $u(0, y) =  y$.  

\vskip .2in 

\underbar{Section 1.6}:  \quad 1, 2, 4

\vskip .2in 
\underbar{Additional Problems:} 

(1) \, Find the general solution of $u_x - \sin(x)\, u_y \,=\,0$. 
Then find the special solution that satisfies $f(0, e^y)= e^{y}$.   

\vskip .1in

(2) \, Find the regions in 2d-space where $x^2\, u_{xx} + 4 u_{xy} +  y^2 \, u_{yy} \, =\,0$ 
is respectively elliptic, parabolic, hyperbolic. Plot these regions. 



\vskip .2in







$\bullet$\, {\bf{Set 8. Due date: 04/18/2019  }}


\vskip .15in

\underbar{Section 4.1}:\quad  2, 3.
\smallskip

\underbar{Hint} In 3) proceed as in the heat equation and  keep the complex $i$ next to $T(t)$.  Recall that the
solution to the  ODE    $T'(t) = i \lambda T(t)$ is  $T(t) = A e^{- i \lambda t} $.

\vskip .1in 

\underbar{Additional Problems} Separate variables to solve the following problems. 
\vskip .1in 

A3.  \quad  $u_{tt} - u_{xx} =0$ in $0<x<3$ with boundary conditions $u(0, t)= u(3, t)=0$

\vskip .1in 

A4. \quad \, $u_t \,=\, u_{xx} $ in $0< x< \pi$ with boundary conditions $u(0,t)\, = \,u(\pi, t) \,=\,0$

\vskip .1in 

A5.  Let $g$ be a smooth function on $[0,1]$, \, $g(0)=g(1) =0$.  Consider the Dirichlet BIVP:
\vskip  .1in
$u_t -  u_{xx} + 3 u= 0\, $ in $0< x< 1$  with $u(0,t)\, = \,u(1, t) \,=\,0$ and  $u(x, 0)= g(x)$  \quad  $(\ast)$
\vskip  .1in
a)  Consider the change of variables $u(x,t) = e^{-3 t} v (x, t)$ and prove that $v$ solves 
\vskip  .1in
$v_t -  v_{xx}= 0$ \, in $0< x< 1$  with $v(0,t)\, = \,v(1, t) \,=\,0$ and $v(x, 0)= g(x)$.   \quad  $(\dagger)$
\vskip  .1in
b)  Use the method of separation of variables to find  $v$,  the solution to $(\dagger)$.
\vskip .1in
c)  Use a) and b) to find $u$, the solution to  $(\ast)$.


\vskip .2in

\underbar{Section 4.2}:\quad 1,  3
\smallskip
\underbar{Hint.}  For 1) read first the last part of section 4.2 (page 91) where there is an example with mixed boundary conditions. 

\vskip .2in

\underbar{Additional Problem 1} Separate variables to solve the following problem for the heat equation with mixed boundary conditions:
$u_t - 5 u_{xx} \,=\, 0$ in $0<x< L$ with boundary conditions $u_x(0,t) = u (L,t)=0$



\vskip .2in 
\bigskip

\underbar{Section 5.1} \, \, \,  2a),  5, 9 
\vskip .1in 
\underbar{Hint for 9:}  Use the trig. identities $\cos (2x) = \cos^2(x) - \sin^2(x) =  2 \cos^2(x) + 1$  to re-express the initial velocity and immediately obtain its cosine "expansion". 

%%
%%   saque de la 5.1  el problema 6
%%

\vskip .1in

\underbar{Section 5.2} \, \, \,  3, 4, 10, 11 

%%   saque el 7 y el 9

\vskip .1in 

\underbar{Additional Problem 2} \,  Let $\phi : \Bbb R \to \Bbb R$ be a periodic function with period $p$; that is $\phi(x + p) = \phi(x), \, \, \forall x \in \Bbb R$. Assume that 
$\phi$ is integrable on any finite interval. 
\smallskip

(a) Prove  that for any $a, b \in \Bbb R$ 
$$  \int_a^b \phi(x) d x  = \int_{a+ p}^{b+p} \phi(x) dx = \int_{a- p}^{b - p} \phi(x) dx $$
\smallskip

(b) Prove  that for any $a \in \Bbb R$ $$  \int_{-p/2}^{p/2} \phi(x+a) d x  = \int_{- p/2 +a }^{p/2+a} \phi(x) dx = \int_{-p/2}^{p/2} \phi(x) dx $$

\smallskip 

Note then  that for any $a \in \Bbb R, \quad  \int_{-p/2}^{p/2} \phi(x+a) d x = \int_{-p/2}^{p/2} \phi(x) dx$ and thus in particular  that $\int_a^{a+p}  \phi(x) \, dx$ does not depend on $a$, as we discussed in class (section 5.2, Strauss). 




\vskip .15in 

\enddocument









\enddocument



                                               














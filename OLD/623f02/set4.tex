\input amstex
\documentstyle{amsppt}
\nologo
\magnification=\magstephalf
\pagewidth{6.5truein}
\pageheight{9truein}
\parskip=5pt
\abovedisplayskip=11pt          %plus 3pt minus 3pt
\belowdisplayskip=11pt          %plus 3pt minus 3ptf

\NoBlackBoxes
\def\card{\operatorname{card}}
\def\intinfin{\int_{-\infty}^{\infty}}
\def\const{\operatorname{const.}}
\def\dist{\operatorname{dist}}
\def\con{\operatorname{const}}
\def\sgn{\operatorname{sgn}}
\def\supp{\operatorname{supp}}
\def\sign{{\rm sgn}}
\def\conphi{{\text const}_{\phi}}
\def\conpsi{{\text const}_{\psi}}
\def\J{{\cal J}}
\def\Q{{\cal Q}}
\def\A{{\Cal A}}
\def\B{{\Cal B}}
\def\C{{\Cal C}}
\def\D{{\Cal D}}
\def\E{{\Cal E}}
\def\F{{\Cal F}}
\def\G{{\Cal G}}
\def\H{{\Cal H}}
\def\I{{\Cal I}}
\def\J{{\Cal J}}
\def\K{{\Cal K}}
\def\L{{\Cal L}}
\def\J{{\Cal J}}
\def\M{{\Cal M}}
\def\Q{{\Cal Q}}
\def\P{{\Cal P}}
\def\S{{\Cal S}}
\def\R{{\Cal R}}
\def\T{{\Cal T}}
\def\V{{\Cal V}}
\def\W{{\Cal W}}
\def\gA{{\frak A}}
\def\gB{{\frak B}}
\def\gF{{\frak F}}
\def\gG{{\frak G}}
\def\gH{{\frak H}}
\def\gN{{\frak N}}
\def\gP{{\frak P}}
\def\gT{{\frak T}}
\def\IC{{\Bbb C}}
\def\II{{\Bbb I}}
\def\IJ{{\Bbb J}}
\def\IN{{\Bbb N}}
\def\IP{{\Bbb P}}
\def\IQ{{\Bbb Q}}
\def\nat{{\Bbb N}}
\def\que{{\Bbb Q}}
\def\IR{{\Bbb R}}
\def\real{{\Bbb R}}
\def\IS{{\Bbb S}}
\def\IT{{\Bbb T}}
\def\IW{{\Bbb W}}
\def\IZ{{\Bbb Z}}
\def\zed{{\Bbb Z}}
%\def\wal#1{\omega_{#1}}
%\def\stof#1{\text{\rm w}_{#1}}
\def\lto{\longrightarrow}
\def\conj{\overline}
\def\sign{\text{sign}}
\def\myskip{\noalign{\vskip6pt}}
%\def\bmatrix#1{\left[ \matrix #1\endmatrix \right]}
\def\varep{\varepsilon}
%\def\ch{\raise 0.3ex\hbox{$\chi$}\kern-.15em}
\def\iprec{\mathop{\prec}_{i}}
\def\3prec{\mathop{\prec}_{3}}
\def\liirr{{L^2_{\rho_r}}}
\def\so{\text{\rm SO}}
\def\bR{\text{\bf R}}
%\def\wal#1{\omega_{#1}}
%\def\stof#1{{\rm w}_{#1}}
%\def\Lip{{\rm Lip}}
\overfullrule=0pt
%\def\ch{\raise 0.3ex\hbox{$\chi$}\kern-.15em}
%\def\real{{\Bbb R}}

\def\dist{\operatorname{dist}}
\def\IH{{\Bbb H}}
\def\gF{{\frak F}}
\def\gN{{\frak N}}
\def\IS{{\Bbb S}}

\topmatter
\title Problem Set  4 \endtitle
\date Due November 7th 2002 \enddate
\endtopmatter

\document

\proclaim{Problem 1}: Do Problem \# 26 in Folland's page 39. Recall that 
$$ E \triangle A := (E \setminus A ) \cup ( A \setminus E) $$
\endproclaim 

\proclaim{Problem 2}: (a) Give the {\it definition} of the outer
measure $m^{\ast}$ that arises in the construction of the Lebesgue
measure $m$ on $\IR$. 

\vskip .1in
(b) Use (a), that is the definition of $m^{\ast}$ to \underbar{prove}
that for any subset $A$ of $\IR$ and any $s \in \IR$, 
$$m^{\ast}( A + s) \ = \ m^{\ast} (A) . $$
\endproclaim 
 
\proclaim{Problem 3 ($\#28$ from Folland p.39)} Let $F$ be increasing and right continuous, and let $\mu_F$ be the associated measure. Then $\mu(\{ a\}) = F(a) - F(a-)$, \, \, $\mu_F([a, b)) = F(b-) - F(a-)$, \, \, $ \mu_F([a, b])= F(b)-F(a-)$, and  $\mu_F((a,b))=F(b-) - F(a)$.
\endproclaim 

\subhead{Remark}\endsubhead  Recall that $$F(a+) = \lim_{ x \searrow a} F(x) = \inf_{ x > a} F(x) \qquad    F(a-) = \lim_{ x \nearrow  a} F(x) = \sup_{ x < a} F(x). $$ Also $F$ is called right continuous if $F(a)= F(a+)$ and left continuous if $F(a)=F(a-)$. 



\proclaim{Problem 4 ($\# 30 $ from Folland p.40)} If $E$ is Lebesgue measurable (i.e. if $E \in \Cal L$ ) and $m(E) >0$, then for any $0< \alpha <1$ there is an open interval $I$ such that $m ( E \cap I) > \alpha \, m(I)$ 
\endproclaim 

\proclaim{Problem 5 ($\# 31 $ from Folland p.40) }  If $E \in \Cal L$ and $m(E) > 0$ the set $E - E := \{ x -y \,  : \, x, y \in E \} $ contains an interval centered at $0$. 

(Hint. Prove that if $I$ is as in the previous exercise with $\alpha > 3/4$ then $E -E$ contains $(- 1/2 m(I), 1/2 m(I))$.  

\endproclaim



\proclaim{Problem 6  ($\# 27 $ from Folland p.39) } Prove that the ternary Cantor set $C$ constructed in class ( see Proposition 1.22 in Folland, pg. 38)  is 
\underbar{compact}, \underbar{nowhere dense} and \underbar{totally
disconnected}.  Prove that moreover, $C$ has no isolated points.

A set $C$ is said to be {\bf totally disconnected} if the only
connected subsets of $C$ are single points. And it is said to be 
{\bf nowhere dense} if $\bar{C}$ has empty interior. 

{\bf Hint.} Show that if $x, y \, \in C$ and $x < y$, there exists a $z
\notin C$ such that $ x < z < y$. 

\endproclaim

\proclaim{Problem 7} Construct a subset of $[0,1]$ in the same manner as the Cantor set, except that at the $kth$ stage, each interval removed has length 
$\delta 3^{-k}$, for some $0 < \delta <1$. Show the resulting set is perfect, has measure $1 - \delta$ and contains no intervals. 

\endproclaim

\proclaim{Problem 8} Let $\{ a_k\}$ be a sequence of real numbers, $ a_k \in (0, 1)$ for all $k \ge 1$. Prove:

$$i) \qquad  \Pi_{k\ge1} \, (1 - a_k)  \, > 0 \, \, \text{ iff } \, \,   \sum_{k\ge1}\, \log (1 - a_k) \, \,  \text{ converges iff } \, \,  \sum_{k \ge 1} a_k   \, \, \text{ converges } $$

ii) \qquad Construct a Cantor-type subset of $[0,1]$ by removing from each interval remaining at the $k-th$ stage a subinterval of relative length $0 < \theta_k <1$. Show that the remainder set ( ie. the infinite intersection of the sets that remained at each stage - as in the Cantor ternary contruction-) has measure zero if and only if $\sum_{k \ge 1} \, \theta_k \,  = \infty $ ( Hint. Use i) ).   

\endproclaim



\enddocument
























\enddocument 

















































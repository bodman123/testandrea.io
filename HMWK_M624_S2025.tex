\magnification=1200
\input amstex

\documentstyle{amsppt}
\NoBlackBoxes
\NoPageNumbers
\pagewidth{16.5truecm}
\pageheight{22truecm}

{\catcode`@=11
\gdef\nologo{\let\logo@\empty}
\catcode`@=12}
\nologo
%\hcorrection{.3in}

\centerline{ \bf  M624 HOMEWORK  -- SPRING 2025}
\vskip .1in
\centerline{ Prof. Andrea R. Nahmod }
\vskip .2in

\document

\head {\underbar{SET 1  - Due 02/13/2025}  }\endhead
\vskip .1in

\noindent {\bf From Chapter  3 (pp 145-146 -Section 5)}: \,  11, 12, 14a), 15, 16a), 23a).


%%%.  14b), 16b), 19, 23b), 32
\medskip


\head {\underbar{SET 2  - Due 02/27/2025}  }\endhead
\vskip .1in


\noindent {\bf From Chapter  3 (pp 145-146 -Section 5)}: \,   14b), 16b), 19, 23b), 32

\bigskip

\noindent {\bf From Chapter 3 (pp 153)}: \,  4 

\bigskip

\noindent\underbar{\bf Additional Questions (Chapter 3)}: After you had read carefully --as assigned in class-- the proofs of Lemma 3.3 and Theorem 3.14 do the following

\vskip .1in

\noindent 1) Explain why $J_{F}(y) - J_{F}(x) \, \leq \, \sum_{\{n \, :\,  x < x_{n} \leq y\}} \, \alpha_{n} \,  \leq  \, F(y) - F(x)$ (proof of Lemma 3.13).
\vskip .1in

\noindent 2) Show rigorously that $J_{F}(x) - F(x)$ is continuous (in proof of Lemma 3.13). 

\vskip .1in
\noindent 3)  Rewrite explaining fully the proof of Theorem 3.14 in Chapter 3. Note you need to solve and use exercise 14 (given above in Chapter 3).
\vskip .1in

\noindent 4)   Justify the step $(\dagger)$ left in class in the proof of Lemma 3.9.  More precisely. Justify why assuming that for $\delta >0$ sufficiently small $m(E) > \delta$ we have that: 
\smallskip

a) For an appropriate compact $E^{\prime} \subseteq E$ we have that $m(E^{\prime}) \geq \delta$  and

\smallskip 

 b) How Lemma 1.2 (first Vitali-type covering Lemma) gives you that from a finite covering of  $E^{\prime}$ (using compactness)  by balls in  $\Cal{B}$ one can select a  \underbar{disjoint} sub-collection of these balls, (call them $B_1, B_2, \dots, B_{N_1}$)   such that  $\sum_{i=1}^{N_1}. m(B_i) \geq 3^{-d} m(E^{\prime}) \geq 3^{-d} \delta$. 
 In other words, justify the two inequalities.

\vskip .2in

\head {\underbar{SET 3  - Due 03/13/2025}  }\endhead
\vskip .1in

%\bigskip
%
%The following problems concern Chapter 3 Section 2 of SS III. Read it and read the 2 Handouts I posted. Then do:
%%\medskip
%
%\noindent {\bf From Chapter  3 (pp 145-146 -Section 5)}: \, 1  [part c) is involved; see handout];  2.

%\medskip
%
%\enddocument
%
%\head {\underbar{SET 3 - Due 03/05/2024}  }\endhead

\vskip .1in 


\noindent {\bf From Chapter 4} :  Read carefully the proofs of both Theorem 2.2 (Riesz-Fisher) on Chaper 2 (p. 70)  and then the one for Theorem 1.2 Chapter 4 (p. 159) from Stein-Shakarchi. 


\medskip 


\noindent {\bf Additional Problem:}  For any $1 \leq p < \infty$ consider the space
$$L^p(\Bbb R^d):= \{ f: \Bbb R^d \to \Bbb C, \text{ measurable}, \, : \, \| f\|_{L^p(\Bbb R^d)}:= \left( \int_{\Bbb R^d}  | f(x) |^p \, dm  \right)^{\frac{1}{p}} < \infty \}.$$
{\underbar{Assume}} that $ \| f\|_{L^p(\Bbb R^d)}$ is a norm$^{\dagger}$  whence $d_p(f,g) : =  \| f - g\|_{L^p(\Bbb R^d)}$ defines a metric and $L^p$ is a metric space.  
Use the proof of completeness that I gave in class from Folland when $p=2$ to  {\bf prove} that $L^p(\Bbb R^d)$  is {\it complete}.
\smallskip

$\dagger$ Aside challenge:  can you guess what would you need to prove the triangle inequality when $p \neq 2$?, 
\bigskip

\noindent {\bf From Chapter 4 (pp 193-194)}:  1, $2^{\ast}$,  3, 4 (show completeness only), 5, 6a),  7,  8a), 10.
\smallskip
$\ast\,\,$ to show that $f-g$ is {\it orthogonal} to $g$ you need to show that $\langle f-g, g \rangle =0$. 

\bigskip

\noindent {\bf Pb.I.} \, Consider $ f \in L^2([-\pi, \pi])$ and \underbar{assume} that $\sum_{n \in \Bbb Z} a_n e^{inx} \,=\, f(x) $ a.e. $x$. Show that on any subinterval $[a, b] \subset [-\pi, \pi] $, 
$$ \int_a^b f(x)\, dx \, = \, \sum_n   \int_a^b  \, a_n \, e^{i n x} \, dx. $$ In particular if $g(x)= \int_a^x f(y) dy $, the Fourier coefficients and series of $g(x)$  can be obtained from $a_n$, the Fourier  coefficients of $f$.

\medskip

{\bf Pb. II.} For $0 < \alpha < 1$, we say that  a function $f$ is  $C^{\alpha}$-H\"older
continuous with exponent  $\alpha$ if there exists a constant $c=c_{\alpha}>0$ such that $|f(x)- f(y)| \leq c\, | x -y|^{\alpha} $ for all $x, y$. 
For $k \in \Bbb N$, we can also define the space $C^{k, \alpha}$ to be that of functions which are $k$-th times differentiable and whose $k$-th derivative is
$C^{\alpha}$-H\"older continuous (we could relabel $C^{\alpha}$ as $C^{0,\alpha}$). 
Consider now $f$ a $2\pi$-periodic $C^{k, \alpha}$ function. If $a_n$ are the Fourier coefficients of $f$,  show that for some $C>0$ independent of $n$, 
$$ |a_n| \leq \frac{C}{|n|^{k+\alpha} } $$

\vskip .1in

\noindent \underbar{\it Bonus Problem}:\,  \,  $2^{\ast}$a)b) from Chapter 4, pp 202.  

\vskip .15in



\head {\underbar{SET 4 - Due 03/27/2025}  }\endhead
\medskip

\noindent {\bf From Chapter 4 (pp 195-197)}: 11, 12, 13 

\medskip

{\bf Pb. I.} \, Consider the subspace $\Cal S$ of $L^2([0,1])$ spanned by the functions: $1 ,\,  x,$ and  $x^3$. 

a) Find an orthonormal basis of $\Cal S$.

b)  Let $P_{\Cal S}$  denote the orthogonal projection on the subspace $\Cal S$, compute $P_{\Cal S} x^2$.


\vskip .1in
{\bf Pb. II.} (This is an undergrad. problem but a very useful property to remember)

\noindent Let $\phi : \Bbb R \to \Bbb R$ be a periodic function with period $p$; that is $\phi(x + p) = \phi(x), \, \, \forall x \in \Bbb R$. Assume that 
$\phi$ is integrable on any finite interval. 
\smallskip

(a) Prove  that for any $a, b \in \Bbb R$ 
$$  \int_a^b \phi(x) d x  = \int_{a+ p}^{b+p} \phi(x) dx = \int_{a- p}^{b - p} \phi(x) dx $$
\smallskip

(b) Prove  that for any $a \in \Bbb R$ $$  \int_{-p/2}^{p/2} \phi(x+a) d x  = \int_{-p/2}^{p/2} \phi(x) dx = \int_{- p/2 +a }^{p/2+a} \phi(x) dx $$

\smallskip 
In particular we have that $\int_a^{a+p}  \phi(x) \, dx$ does not depend on $a$, as we discussed in class. 

\vskip .15in

\noindent {\bf Assigned Reading  from Chapter  4 of  [Stein-Shakarchi Vol 3]:} 

\medskip
a)  Examples 1) and 2). on pages 178--180. 

\bigskip
b) We will cover subsection ections 4.5.1 and 4.5.2 of section 5 chapter 4 in class but \underbar{subsection 4.5.3  pages 185--188} \underbar{is assigned reading}. It is an important subsection. Please read ahead!

\vskip .15in

\head {\underbar{SET 5 - Due 04/03/2025 }  }\endhead
\vskip  .2in

\noindent {\bf From Chapter 4 (pp 195-197)}:   18, 19, 20

\medskip

\noindent {\bf Assigned Reading (in class) from Chapter  4 of  [Stein-Shakarchi Vol 3]:} 

Remarks  (a), (b), (c) on pages 184-185. 

\medskip
\noindent \underbar{{\bf Then turn in:}}



\smallskip
\noindent {\bf From Chapter 4 (pp 187)}: Read and rewrite filling in all details the Proof of Proposition 5.5.

\bigskip

\noindent {\bf From Chapter 4 (pp 189)}: Read and rewrite filling in all details the Proof of  Proposition 6.1.

\bigskip

\noindent {\bf From Chapter 4 (pp 197-202)}:   21, 22, 23, 25, 26, 28. 


\vskip .15in

\head {\underbar{SET 6 - Due 04/10/2025}}  \endhead
\vskip  .2in

\noindent {\bf From Chapter 4 (pp 197-202)}:   30,  32, 33.
\bigskip

\noindent  \underbar{\it Bonus Problems }:  $29^{\ast}$ (p 199-200) and $6^{\ast}$ (p. 203-204). These are about Fredholm's Alternative for compact operators.
\vskip  .2in


\head {\underbar{SET 7 - Due 05/08/2025 }  }\endhead
\vskip  .2in
\noindent {\bf From Chapter 5 (pp 253-255)}:   1, 9. See definition, examples and hints below.
\bigskip

\noindent \underbar{Definition:}  A Fourier multiplier operator $T$ on $\Bbb R^d$ is a linear operator on $L^2(\Bbb R^d)$ determined by a bounded function $m$ (the multiplier) such that $T$ is defined by the formula
$$ \widehat{T(f)}(\xi):= m(\xi) \widehat{f}(\xi) $$ for all $\xi \in \Bbb R^d$ and any $f \in L^2(\Bbb R^d).$ 

\smallskip

\noindent \underbar{Examples}. The bounded linear operator  $P_N : L^2 (\Bbb R) \to  L^2 (\Bbb R)$ defined by  $\widehat{P_N(f)}(\xi) : = \chi_{[-N, N]} (\xi) \, \widehat{f}(\xi)$  is one such operator. In fact is an orthogonal projection. 
\smallskip
Another well known one is the {\it Hilbert Transform} $\Cal H :L^2(\Bbb R) \to L^2(\Bbb R)$ 
defined by $\widehat{\Cal{H}(f)}(\xi) : =  - i  sgn(\xi) \, \widehat{f}(\xi)$. The operator $\Cal{H}$ is bounded and linear on $L^2$.  That is bounded foillows from the fact that $ - i  sgn(\xi) $ is bounded point-wise by 1 and Theorem 1.1 that says the Fourier transform  is unitary on $L^2(\Bbb R^d)$

\bigskip 


\underbar{Hints for 1)} Note that a) follows from b) since if a function is in  $L^2$ then it is finite  a.e. (in fact you get that $\int |f(x-y) k(y)| dy$ is finite a.e. $x$.) so prove b) first.
Then for c) first prove it for functions in $L^1 \cap L^2$. Then use a density argument:  given  $f \in L^2$  assume  the sequence $f_n \in L^1 \cap L^2$ approximates $f $ in the $L^2$ sense ( $L^2$-norm) . Then bound
$ | \widehat{f \ast k} (\xi)  -  \widehat{f_n \ast k}(\xi) |$ by $\| f_n -f\|_{L^2} \|k\|_{L^1}$ and conclude from here. Finally part d) is the definition of a Fourier multiplication operator applied to part c)

\bigskip 

\noindent {\bf Additional Problem} \,Carefully read and rewrite on your own (justifying and filling the gaps as necessary) Lemma 1.2 on page 209 proving that ${\Cal S}(\Bbb R^d)$ is dense in $L^2(\Bbb R^d)$
\bigskip

\noindent {\bf From Chapter 6}:  Read/Study the proofs in Section 1.

\vskip .2in

\noindent {\bf From Chapter 6 (pp 312 313)}:  1 (change $\Cal M$ to be a non-empty algebra); 10.  

\bigskip

\enddocument


%%%%%%%%%%%%%%%%%%%%%%%%%%%%%%%%%%%%%%%%%%%%%


\head {\underbar{SET 6 - Due 04/18/2024 }  }\endhead
\vskip  .2in
\noindent {\bf From Chapter 5 (pp 253-255)}:   1, 9 (see definition and example below).
\bigskip

\noindent \underbar{Definition:}  A Fourier multiplier operator $T$ on $\Bbb R^d$ is a linear operator on $L^2(\Bbb R^d)$ determined by a bounded function $m$ (the multiplier) such that $T$ is defined by the formula
$$ \widehat{T(f)}(\xi):= m(\xi) \widehat{f}(\xi) $$ for all $\xi \in \Bbb R^d$ and any $f \in L^2(\Bbb R^d).$ 

\smallskip

\noindent \underbar{Examples}. The bounded linear operator  $P_N : L^2 (\Bbb R) \to  L^2 (\Bbb R)$ defined by  $\widehat{P_N(f)}(\xi) : = \chi_{[-N, N]} (\xi) \, \widehat{f}(\xi)$  is one such operator. In fact is an orthogonal projection. 
\smallskip
Another well known one is the {\it Hilbert Transform} $\Cal H :L^2(\Bbb R) \to L^2(\Bbb R)$ 
defined by $\widehat{\Cal{H}(f)}(\xi) : =  - i  sgn(\xi) \, \widehat{f}(\xi)$. The operator $\Cal{H}$ is bounded and linear on $L^2$.  That is bounded foillows from the fact that $ - i  sgn(\xi) $ is bounded point-wise by 1 and Theorem 1.1 that says the Fourier transform  is unitary on $L^2(\Bbb R^d)$


\bigskip 

\noindent {\bf From Chapter 5 (pp 260-261)}:   6.
\bigskip

\noindent {\bf Additional Problem} \,Carefully read and rewrite on your own (justifying and filling the gaps as necessary) Lemma 1.2 on page 209 proving that ${\Cal S}(\Bbb R^d)$ is dense in $L^2(\Bbb R^d)$
\bigskip

\noindent {\bf From Chapter 6}:  Read/Study the proofs in Section 1.

\vskip .2in

\noindent {\bf From Chapter 6 (pp 312 313)}:    1 (change $\Cal M$ to be a non-empty algebra), 2a), 8.  

\bigskip



\noindent {\underbar{\it Extra Problem}} (do not to turn in): \noindent {\bf From Chapter 5 (pp 260-261)}:  5
\smallskip 
%
%\underbar{Definition:}  A Fourier multiplier operator $T$ on $\Bbb R^d$ is a linear operator on $L^2(\Bbb R^d)$ determined by a bounded function $m$ (the multiplier) such that $T$ is defined by the formula
%$$ \widehat{T(f)}(\xi):= m(\xi) \widehat{f}(\xi) $$ for all $\xi \in \Bbb R^d$ and any $f \in L^2(\Bbb R^d).$ 
%
%\smallskip
%
%\underbar{Examples}. The bounded linear operator  $P_N : L^2 (\Bbb R) \to  L^2 (\Bbb R)$ defined by  $\widehat{P_N(f)}(\xi) : = \chi_{[-N, N]} (\xi) \, \widehat{f}(\xi)$  is one such operator. In fact is an orthogonal projection. 
%\smallskip
%Another well known one is the {\it Hilbert Transform} $\Cal H :L^2(\Bbb R) \to L^2(\Bbb R)$ 
%defined by $\widehat{\Cal{H}(f)}(\xi) : =  - i  sgn(\xi) \, \widehat{f}(\xi)$. The operator $\Cal{H}$ is bounded and linear on $L^2$. 
%
\vskip .2in

\head {\underbar{SET 7 - Due 04/25/2024}}  \endhead



\noindent {\bf From Chapter 6 (pp 317-322)}:   5, 10, 11a)b), 16a)b)

\bigskip

\underbar{Additional Problems:}  

\smallskip

\noindent {\bf(A1)} \, Let $\nu$ be a finite signed measure on $(X, \Cal M)$. Show that for any $E \in \Cal M$

$$ \align |\nu|(E) &= \\
&= \sup \{ \sum_{k=1}^K\,  |\nu(E_k)| \, : \,  E_1, \dots E_K  \text{ are disjoint and }\,  E=\cup_{k=1}^K E_k  \, \} \tag{1} \\
&=   \sup \{ \, \sum_{k=1}^{\infty}\,  |\nu(E_k)| \, : \,  E_1,  E_2,  \dots   \text{ are disjoint and } \,  E=\cup_{k=1}^{\infty} E_k \, \}  \tag{2} \\ 
&=   \sup \{ | \int_{E}   f  d \nu |  \, : \,  |f|  \leq 1 \}  \tag{3}
\endalign $$ 

You may want to proceed for example by proving that  $(1) \leq (2) \leq (3) \leq (1)$. 

\bigskip

\noindent {\bf(A2)} \, Let $F \in BV([a,b])$ and right continuous. Let $G(x)= |\mu_F|([a, x])$. Show that $|\mu_F|=\mu_{T_F}$ by showing that $G=T_F$. To do so you may proceed by proving:

\smallskip

1) \,  $T_F  \leq G$  (use definition of $T_F$).
\smallskip

2) \, $|\mu_F(E)| \leq \mu_{T_F} (E) $ for any Borel set $E$  (do for an interval first).

\smallskip

3)  Show that  $|\mu_F| \leq \mu_{T_F}$ and hence $G \leq T_F$  (use (A1)). 

\vskip .3in
\underbar{{\bf Do} (but do not turn in)}:  \,9, 16c)d)e)f). 



%\enddocument
\vskip .2in



\head {\underbar{SET 8 - Due together with Final Exam on 05/15/2024}}  \endhead


%\underbar{Additional Problems:}  

%\smallskip
%
%\noindent {\bf(A1)} \, Let $\nu$ be a finite signed measure on $(X, \Cal M)$. Show that for any $E \in \Cal M$
%
%$$ \align |\nu|(E) &= \\
%&= \sup \{ \sum_{k=1}^K\,  |\nu(E_k)| \, : \,  E_1, \dots E_K  \text{ are disjoint and }\,  E=\cup_{k=1}^K E_k  \, \} \tag{1} \\
%&=   \sup \{ \, \sum_{k=1}^{\infty}\,  |\nu(E_k)| \, : \,  E_1,  E_2,  \dots   \text{ are disjoint and } \,  E=\cup_{k=1}^{\infty} E_k \, \}  \tag{2} \\ 
%&=   \sup \{ | \int_{E}   f  d \nu |  \, : \,  |f|  \leq 1 \}  \tag{3}
%\endalign $$ 
%
%You may want to proceed for example by proving that  $(1) \leq (2) \leq (3) \leq (1)$. 
%
%\medskip

%\noindent {\bf(A2)} \, Let $F \in BV([a,b])$ and right continuous. Let $G(x)= |\mu_F|([a, x])$. Show that $|\mu_F|=\mu_{T_F}$ by showing that $G=T_F$. To do so you may proceed by proving:
%
%1) \,  $T_F  \leq G$  (use definition of $T_F$).
%
%2) \, $|\mu_F(E)| \leq \mu_{T_F} (E) $ for any Borel set $E$  (do for an interval first).
%
%3)  Show that  $|\mu_F| \leq \mu_{T_F}$ and hence $G \leq T_F$  (use (A1)). 
%
%\bigskip
%\underbar{{\bf Do} (but do not turn in)}:  \,9, 16c)d)e)f). 
%
%\vskip .3in
%\head {\underbar{SET 9 - Due together with Final Exam 05/08/18}}  \endhead
%
%\bigskip

\noindent {\bf From Chapter 1 of [SS, Vol. 4] (pp 34-43)}:  1a)b),  5, 8, 12a), 13a)b), 15, 16, 17a)c).
\smallskip




\underbar{Hint} For 12 use the Riesz representation theorem. 
\smallskip

\underbar{Hint} For 16 use the known H\"older for two functions and then induction.

\bigskip

\underbar{{\bf Do} (but do not turn in)}:   6,\,7, 19, 20. 
\enddocument




\vskip .2in
\head {\underbar{SET 7 - Thursday April 20th}}  \endhead



\noindent {\bf From Chapter 6 (pp 317-322)}:   8,   ( 9?) \, 10, 11a)b), 16a)b)

\medskip

\underbar{Additional Problems:}  

\smallskip

\noindent {\bf(A1)} \, Let $\nu$ be a signed measure on $(X, \Cal M)$. Show that for any $E \in \Cal M$

$$ \align |\nu|(E) &= \\
&= \sup \{ \sum_{k=1}^K\,  |\nu(E_k)| \, : \,  E_1, \dots E_K  \text{ are disjoint and }\,  E=\cup_{k=1}^K E_k  \, \} \tag{1} \\
&=   \sup \{ \, \sum_{k=1}^{\infty}\,  |\nu(E_k)| \, : \,  E_1,  E_2,  \dots   \text{ are disjoint and } \,  E=\cup_{k=1}^{\infty} E_k \, \}  \tag{2} \\ 
&=   \sup \{ | \int_{E}   f  d \nu |  \, : \,  |f|  \leq 1 \}  \tag{3}
\endalign $$ 

You may want to proceed for example by proving that  $(1) \leq (2) \leq (3) \leq (1)$. 

\medskip

\noindent {\bf(A2)} \, Let $F \in BV([a,b])$ and right continuous. Let $G(x)= |\mu_F|([a, x])$. Show that $|\mu_F|=\mu_{T_F}$ by showing that $G=T_F$. To do so you may proceed by proving:

1) \,  $T_F  \leq G$  (use definition of $T_F$).

2) \, $|\mu_F(E)| \leq \mu_{T_F} (E) $ for any Borel set $E$  (do for an interval first).

3)  Show that  $|\mu_F| \leq \mu_{T_F}$ and hence $G \leq T_F$  (use (A1)). 

\medskip
\underbar{Problems {\bf to do} (but do not turn in)}:  \, 14, 16c)d)e)f). 

\vskip .1in

\bigskip

\noindent {\bf From Chapter 1 of [SS, Vol. 4] (pp 34-43)}:  1, 3, 5, 6, 7, 8.



%\noindent {\bf(A3)}  Let $F$ and $G$ be $BV([a,b])$ and right continuous. Let $\mu_F$ and $\mu_G$ be the corresponding signed Borel measures
%(recall these measures are uniquely determined by -say-  $\mu_F( c, d])= F(d) - F(c)$).  
%\smallskip
%
%a)  Show that if either $F$ or $G$ are continuous the following {\it integration by parts} formula holds:
%$$ \int_{(a, b]} F d\mu_G + \int_{(a, b]} G d\mu_F \, =\, F(b)G(b) - F(a)G(a) $$
%
%\smallskip
%b)  If  $F$ and $G$ are absolutely continuous then 
%$$ \int_{(a, b]} F G^{\prime} \, dx \,+\,  \int_{(a, b]} G F^{\prime} \, dx \, =\, F(b)G(b) - F(a)G(a) $$

 
\vskip .2in

 
 \head {\underbar{SET 8 - Due 04/27/15}}\endhead
\medskip

\noindent {\bf From Chapter 1 of [SS, Vol. 4] (pp 36-43)}: 9,  12 (do this on $\Bbb R^n$ with Lebesgue measure), 13, 15, 16, 17, 19, 20, 34, 35.




\end{document}




\medskip

\noindent {\bf(A3)}  Let $F$ and $G$ be $BV([a,b])$ and right continuous. Let $\mu_F$ and $\mu_G$ be the corresponding signed Borel measures
(recall these measures are uniquely determined by -say-  $\mu_F( c, d])= F(d) - F(c)$).  
\smallskip

a)  Show that if either $F$ or $G$ are continuous the following {\it integration by parts} formula holds:
$$ \int_{(a, b]} F d\mu_G + \int_{(a, b]} G d\mu_F \, =\, F(b)G(b) - F(a)G(a) $$

\smallskip
b)  If  $F$ and $G$ are absolutely continuous then 
$$ \int_{(a, b]} F G^{\prime} \, dx \,+\,  \int_{(a, b]} G F^{\prime} \, dx \, =\, F(b)G(b) - F(a)G(a) $$

\bigskip

\underbar{Problems {\bf to do} (but do not turn in)}:  \, 14, 16c)d)e)f) 
 
 \bigskip
 
 \bigskip
 
 \head {\underbar{SET 7 - Due 04/16/15 }  }\endhead
\medskip

\noindent {\bf From Chapter 1 of [SS, Vol. 4] (pp 34-43)}:  1, 3, 5, 6, 7, 8.


\head {\underbar{SET 8 - Due 04/30/15}}\endhead
\medskip

\noindent {\bf From Chapter 1 of [SS, Vol. 4] (pp 36-43)}: 9,  12 (do this on $\Bbb R^n$ with Lebesgue measure), 13, 15, 16, 17, 19, 20, 34, 35.


%\noindent {\bf Additional Problem:}  For any $1 \leq p < \infty$ consider the space
%$$L^p(\Bbb R^d):= \{ f: \Bbb R^d \to \Bbb C, \text{ measurable}, \, : \, \| f\|_{L^p(\Bbb R^d)}:= \left( \int_{\Bbb R^d}  | f(x) |^p \, dm  \right)^{\frac{1}{p}} < \infty \}.$$
%{\underbar{Assume}} that $ \| f\|_{L^p(\Bbb R^d)}$ is a norm ( {\it challenge:} can you guess what would you need to prove the triangle inequality when $p \neq 2$?), whence $d_p(f,g) : =  \| f - g\|_{L^p(\Bbb R^d)}$ defines a metric and $L^p$ is a metric space.  {\bf Prove} that $L^p(\Bbb R^d)$  is {\it complete}.
%
 
 \enddocument



                                               














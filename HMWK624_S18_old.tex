\magnification=1200
\input amstex

\documentstyle{amsppt}
\NoBlackBoxes
\NoPageNumbers
\pagewidth{16.5truecm}
\pageheight{22truecm}

{\catcode`@=11
\gdef\nologo{\let\logo@\empty}
\catcode`@=12}
\nologo
%\hcorrection{.3in}

\centerline{ \bf  M624 HOMEWORK  -- SPRING 2018}
\vskip .1in
\centerline{ Prof. Andrea R. Nahmod }
\vskip .2in

\document

\head {\underbar{SETS 1\,\&\,2 - Due 02/08/2018}  }\endhead
\vskip .1in

\noindent {\bf From Chapter  3 (pp 145-146 -Section 5)}: \, 4, 5, 7, 11, 12, 14, 15, 16, 19, 23, 32.

\medskip
\noindent {\bf From Chapter 3 (pp 153)}:  4.

\bigskip
\noindent {\bf From Chapter  3 (pp 152- Section 6) Bonus Problem}: \,  1  \,(this problem is based on a good understanding of Lemma 1.2  p.102 and of Lemma 3.9 p.128-definition of Vitali covering is just before the statement of Lemma 3.9.)
\vskip .2in

\noindent\underbar{\bf Additional Questions (Chapter 3; left in class)}:

\vskip .1in

\noindent 1) Explain why $J_{F}(y) - J_{F}(x)  \leq \sum_{n : x < x_{n} \leq y} \alpha_{n}  \leq  F(y) - F(x)$ (proof of Lemma 3.13).
\vskip .1in

\noindent 2) Show rigorously that $J_{F}(x) - F(x)$ is continuous (in proof of Lemma 3.13). 

\vskip .1in
\noindent 3)  Rewrite explaining fully the proof of Theorem 3.14 in Chapter 3. Note you need to solve and use exercise 14 (given above in Chapter 3).


\head {\underbar{SET 3 - Due 02/15/18}  }\endhead

\vskip .1in 


\noindent {\bf From Chapter  3 (pp 145-146 -Section 5)}: \, 1  (part c) is involved; see handout)
\bigskip

\noindent {\bf From Chapter 4} :  Recall carefully the proof of both  Theorem 2.2 (Riesz-Fisher) on Chaper 2 (p. 70)  and then the one for Theorem 1.2 Chapter 4 (p. 159) 

%\noindent {\bf Additional Problem:}  For any $1 \leq p < \infty$ consider the space
%$$L^p(\Bbb R^d):= \{ f: \Bbb R^d \to \Bbb C, \text{ measurable}, \, : \, \| f\|_{L^p(\Bbb R^d)}:= \left( \int_{\Bbb R^d}  | f(x) |^p \, dm  \right)^{\frac{1}{p}} < \infty \}.$$
%{\underbar{Assume}} that $ \| f\|_{L^p(\Bbb R^d)}$ is a norm ( {\it challenge:} can you guess what would you need to prove the triangle inequality when $p \neq 2$?), whence $d_p(f,g) : =  \| f - g\|_{L^p(\Bbb R^d)}$ defines a metric and $L^p$ is a metric space.  {\bf Prove} that $L^p(\Bbb R^d)$  is {\it complete}.
%
\bigskip

\noindent {\bf From Chapter 4 (pp 193-194)}:  $2^{\ast}$,  5.
\smallskip
$\ast\,\,$ to show that $f-g$ is {\it orthogonal} to $g$ you need to show that $\langle f-g, g \rangle =0$. 


\head {\underbar{SET 4 - Due 03/01/18}  }\endhead
\vskip .1in 

\noindent {\bf From Chapter 4 (pp 193-194)}:  1, 3, 4 (show completeness only), 6a),  7,  8a), 10.

\medskip

\noindent {\bf Pb.I.} \, Consider $ f \in L^2([-\pi, \pi])$ and \underbar{assume} that $\sum_{n \in \Bbb Z} a_n e^{inx} \,=\, f(x) $ a.e. $x$. Show that on any subinterval $[a, b] \subset [-\pi, \pi] $, 
$$ \int_a^b f(x)\, dx \, = \, \sum_n   \int_a^b  \, a_n \, e^{i n x} \, dx. $$ In particular if $g(x)= \int_a^x f(y) dy $, the Fourier coefficients and series of $g(x)$  can be obtained from $a_n$, the Fourier  coefficients of $f$.

\medskip

{\bf Pb. II.} For $0 < \alpha < 1$, we say that  a function $f$ is  $C^{\alpha}$-H\"older
continuous with exponent  $\alpha$ if there exists a constant $c=c_{\alpha}>0$ such that $|f(x)- f(y)| \leq c\, | x -y|^{\alpha} $ for all $x, y$. 
For $k \in \Bbb N$, we can also define the space $C^{k, \alpha}$ to be that of functions which are $k$-th times differentiable and whose $k$-th derivative is
$C^{\alpha}$-H\"older continuous (we could relabel $C^{\alpha}$ as $C^{0,\alpha}$). 

Consider now $f$ a $2\pi$-periodic $C^{k, \alpha}$ function. If $a_n$ are the Fourier coefficients of $f$,  show that for some $C>0$ independent of $n$, 
$$ |a_n| \leq \frac{C}{|n|^{k+\alpha} } $$

\vskip .1in

\noindent \underbar{\it Bonus Problem}:\,  \,  $2^{\ast}$a)b) from Chapter 4, pp 202. 


\head {\underbar{SET 5 - Due 03/08/18}  }\endhead

\noindent {\bf From Chapter 4 (pp 195-197)}: 11, 12, 13, 20

\medskip

{\bf Pb. I.} \, Consider the subspace $\Cal S$ of $L^2([0,1])$ spanned by the functions: $1 ,\,  x,$ and  $x^3$. 

a) Find an orthonormal basis of $\Cal S$.

b)  Let $P_{\Cal S}$  denote the orthogonal projection on the subspace $\Cal S$, compute $P_{\Cal S} x^2$.


\vskip .1in
{\bf Pb. II.} \,  Let $\phi : \Bbb R \to \Bbb R$ be a periodic function with period $p$; that is $\phi(x + p) = \phi(x), \, \, \forall x \in \Bbb R$. Assume that 
$\phi$ is integrable on any finite interval. 
\smallskip

(a) Prove  that for any $a, b \in \Bbb R$ 
$$  \int_a^b \phi(x) d x  = \int_{a+ p}^{b+p} \phi(x) dx = \int_{a- p}^{b - p} \phi(x) dx $$
\smallskip

(b) Prove  that for any $a \in \Bbb R$ $$  \int_{-p/2}^{p/2} \phi(x+a) d x  = \int_{-p/2}^{p/2} \phi(x) dx = \int_{- p/2 +a }^{p/2+a} \phi(x) dx $$

\smallskip 
In particular we have that $\int_a^{a+p}  \phi(x) \, dx$ does not depend on $a$, as we discussed in class. 
\vskip  .2in



\noindent {\bf From Chapter 4 (pp 197-202)}:  18, 19, 21a), 22, 26.


\vskip .1in

\noindent  \underbar{\it Bonus Problem}:   11*a)b)  from chapter 4, pp 205.

\vskip .2in

\head {\underbar{SET 6 - Due 03/22/18 }  }\endhead
\vskip .1in
\noindent {\bf From Chapter 4 (pp 196-202)}:  21b), 23, 25, 28, 30, 32, 33.
\bigskip
\noindent  \underbar{\it Bonus Problems}:  29 (p 199-200) and $6^{\ast}$ (p. 203-204). These are about Fredholm's Alternative for compact operators.



\vskip .2in

\head {\underbar{SET 7 - Due 04/12/18}}  \endhead

\noindent {\bf From Chapter 5 (pp 253-255)}:   1, 9.

\underbar{Definition:}  A Fourier multiplier operator $T$ on $\Bbb R^d$ is a linear operator on $L^2(\Bbb R^d)$ determined by a bounded function $m$ (the multiplier) such that $T$ is defined by the formula
$$ \widehat{T(f)}(\xi):= m(\xi) \widehat{f}(\xi) $$ for all $\xi \in \Bbb R^d$ and any $f \in L^2(\Bbb R^d).$ 

\bigskip

The bounded linear operator  $P_N : L^2 (\Bbb R) \to  L^2 (\Bbb R)$ defined by  $\widehat{P_N(f)}(\xi) : = \chi_{[-N, N]} (\xi) \, \widehat{f}(\xi)$  is one such operator. In fact is an orthogonal projection. 
\smallskip

Another well known one is the {\it Hilbert Transform} $\Cal H :L^2(\Bbb R) \to L^2(\Bbb R)$ 
defined by $\widehat{\Cal{H}(f)}(\xi) : =  - i  sgn(\xi) \, \widehat{f}(\xi)$. The operator $\Cal{H}$ is bounded and linear on $L^2$. 

\bigskip 
\noindent {\bf From Chapter 5 (pp 260)}:   5. 

\bigskip 

\noindent {\bf From Chapter 6}:  Read/Study the proofs in Section 1.

\bigskip

\noindent {\bf From Chapter 6}:  1 (one should start with an algebra), 2a), 3.

\bigskip

\noindent {\bf From Chapter 6 (pp 313)}:  5.


\head {\underbar{SET 8 - Due 05/01/18}}  \endhead



\noindent {\bf From Chapter 6 (pp 317-322)}:   8,  10, 11a)b), 16a)b)

\medskip

\underbar{Additional Problems:}  

\smallskip

\noindent {\bf(A1)} \, Let $\nu$ be a finite signed measure on $(X, \Cal M)$. Show that for any $E \in \Cal M$

$$ \align |\nu|(E) &= \\
&= \sup \{ \sum_{k=1}^K\,  |\nu(E_k)| \, : \,  E_1, \dots E_K  \text{ are disjoint and }\,  E=\cup_{k=1}^K E_k  \, \} \tag{1} \\
&=   \sup \{ \, \sum_{k=1}^{\infty}\,  |\nu(E_k)| \, : \,  E_1,  E_2,  \dots   \text{ are disjoint and } \,  E=\cup_{k=1}^{\infty} E_k \, \}  \tag{2} \\ 
&=   \sup \{ | \int_{E}   f  d \nu |  \, : \,  |f|  \leq 1 \}  \tag{3}
\endalign $$ 

You may want to proceed for example by proving that  $(1) \leq (2) \leq (3) \leq (1)$. 

\medskip

\noindent {\bf(A2)} \, Let $F \in BV([a,b])$ and right continuous. Let $G(x)= |\mu_F|([a, x])$. Show that $|\mu_F|=\mu_{T_F}$ by showing that $G=T_F$. To do so you may proceed by proving:

1) \,  $T_F  \leq G$  (use definition of $T_F$).

2) \, $|\mu_F(E)| \leq \mu_{T_F} (E) $ for any Borel set $E$  (do for an interval first).

3)  Show that  $|\mu_F| \leq \mu_{T_F}$ and hence $G \leq T_F$  (use (A1)). 

\bigskip
\underbar{{\bf Do} (but do not turn in)}:  \,9, 16c)d)e)f). 

\vskip .3in
\head {\underbar{SET 9 - Due together with Final Exam 05/08/18}}  \endhead

\bigskip

\noindent {\bf From Chapter 1 of [SS, Vol. 4] (pp 34-43)}:  5, 8, 13, 15, 16, 17.
\smallskip

\underbar{Hint} For 16 use the known H\"older for two functions and then induction.
\bigskip

\underbar{{\bf Do} (but do not turn in)}:   6,\,7, 12, 19, 20, 21. 
\smallskip

\underbar{Hint} For 12 use the Riesz representation theorem. 
\enddocument




\vskip .2in
\head {\underbar{SET 7 - Thursday April 20th}}  \endhead



\noindent {\bf From Chapter 6 (pp 317-322)}:   8,   ( 9?) \, 10, 11a)b), 16a)b)

\medskip

\underbar{Additional Problems:}  

\smallskip

\noindent {\bf(A1)} \, Let $\nu$ be a signed measure on $(X, \Cal M)$. Show that for any $E \in \Cal M$

$$ \align |\nu|(E) &= \\
&= \sup \{ \sum_{k=1}^K\,  |\nu(E_k)| \, : \,  E_1, \dots E_K  \text{ are disjoint and }\,  E=\cup_{k=1}^K E_k  \, \} \tag{1} \\
&=   \sup \{ \, \sum_{k=1}^{\infty}\,  |\nu(E_k)| \, : \,  E_1,  E_2,  \dots   \text{ are disjoint and } \,  E=\cup_{k=1}^{\infty} E_k \, \}  \tag{2} \\ 
&=   \sup \{ | \int_{E}   f  d \nu |  \, : \,  |f|  \leq 1 \}  \tag{3}
\endalign $$ 

You may want to proceed for example by proving that  $(1) \leq (2) \leq (3) \leq (1)$. 

\medskip

\noindent {\bf(A2)} \, Let $F \in BV([a,b])$ and right continuous. Let $G(x)= |\mu_F|([a, x])$. Show that $|\mu_F|=\mu_{T_F}$ by showing that $G=T_F$. To do so you may proceed by proving:

1) \,  $T_F  \leq G$  (use definition of $T_F$).

2) \, $|\mu_F(E)| \leq \mu_{T_F} (E) $ for any Borel set $E$  (do for an interval first).

3)  Show that  $|\mu_F| \leq \mu_{T_F}$ and hence $G \leq T_F$  (use (A1)). 

\medskip
\underbar{Problems {\bf to do} (but do not turn in)}:  \, 14, 16c)d)e)f). 

\vskip .1in

\bigskip

\noindent {\bf From Chapter 1 of [SS, Vol. 4] (pp 34-43)}:  1, 3, 5, 6, 7, 8.



%\noindent {\bf(A3)}  Let $F$ and $G$ be $BV([a,b])$ and right continuous. Let $\mu_F$ and $\mu_G$ be the corresponding signed Borel measures
%(recall these measures are uniquely determined by -say-  $\mu_F( c, d])= F(d) - F(c)$).  
%\smallskip
%
%a)  Show that if either $F$ or $G$ are continuous the following {\it integration by parts} formula holds:
%$$ \int_{(a, b]} F d\mu_G + \int_{(a, b]} G d\mu_F \, =\, F(b)G(b) - F(a)G(a) $$
%
%\smallskip
%b)  If  $F$ and $G$ are absolutely continuous then 
%$$ \int_{(a, b]} F G^{\prime} \, dx \,+\,  \int_{(a, b]} G F^{\prime} \, dx \, =\, F(b)G(b) - F(a)G(a) $$

 
\vskip .2in

 
 \head {\underbar{SET 8 - Due 04/27/15}}\endhead
\medskip

\noindent {\bf From Chapter 1 of [SS, Vol. 4] (pp 36-43)}: 9,  12 (do this on $\Bbb R^n$ with Lebesgue measure), 13, 15, 16, 17, 19, 20, 34, 35.




\end{document}




\medskip

\noindent {\bf(A3)}  Let $F$ and $G$ be $BV([a,b])$ and right continuous. Let $\mu_F$ and $\mu_G$ be the corresponding signed Borel measures
(recall these measures are uniquely determined by -say-  $\mu_F( c, d])= F(d) - F(c)$).  
\smallskip

a)  Show that if either $F$ or $G$ are continuous the following {\it integration by parts} formula holds:
$$ \int_{(a, b]} F d\mu_G + \int_{(a, b]} G d\mu_F \, =\, F(b)G(b) - F(a)G(a) $$

\smallskip
b)  If  $F$ and $G$ are absolutely continuous then 
$$ \int_{(a, b]} F G^{\prime} \, dx \,+\,  \int_{(a, b]} G F^{\prime} \, dx \, =\, F(b)G(b) - F(a)G(a) $$

\bigskip

\underbar{Problems {\bf to do} (but do not turn in)}:  \, 14, 16c)d)e)f) 
 
 \bigskip
 
 \bigskip
 
 \head {\underbar{SET 7 - Due 04/16/15 }  }\endhead
\medskip

\noindent {\bf From Chapter 1 of [SS, Vol. 4] (pp 34-43)}:  1, 3, 5, 6, 7, 8.


\head {\underbar{SET 8 - Due 04/30/15}}\endhead
\medskip

\noindent {\bf From Chapter 1 of [SS, Vol. 4] (pp 36-43)}: 9,  12 (do this on $\Bbb R^n$ with Lebesgue measure), 13, 15, 16, 17, 19, 20, 34, 35.


%\noindent {\bf Additional Problem:}  For any $1 \leq p < \infty$ consider the space
%$$L^p(\Bbb R^d):= \{ f: \Bbb R^d \to \Bbb C, \text{ measurable}, \, : \, \| f\|_{L^p(\Bbb R^d)}:= \left( \int_{\Bbb R^d}  | f(x) |^p \, dm  \right)^{\frac{1}{p}} < \infty \}.$$
%{\underbar{Assume}} that $ \| f\|_{L^p(\Bbb R^d)}$ is a norm ( {\it challenge:} can you guess what would you need to prove the triangle inequality when $p \neq 2$?), whence $d_p(f,g) : =  \| f - g\|_{L^p(\Bbb R^d)}$ defines a metric and $L^p$ is a metric space.  {\bf Prove} that $L^p(\Bbb R^d)$  is {\it complete}.
%
 
 \enddocument



                                               













